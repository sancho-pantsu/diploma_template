\documentclass[
bachelor, % document type
subf, % use and configure subfig package for nested figure numbering
times % use Times font as main
]{disser}

% Кодировка и язык
\usepackage[T2A]{fontenc} % поддержка кириллицы
\usepackage[utf8]{inputenc} % кодировка исходного текста
\usepackage[english,russian]{babel} % переключение языков

% Геометрия страницы и графика
\usepackage[left=3cm, right=1.5cm, top=2cm, bottom=2cm]{geometry} % поля страницы
\usepackage{graphicx} % подключение графики
\usepackage{pdfpages} % вставка pdf-страниц

% Таблицы
\usepackage{array} % расширенные возможности для работы с таблицами
\usepackage{tabularx} % автоматический подбор ширины столбцов
\usepackage{dcolumn} % выравнивание чисел по разделителю

% Библиография и ссылки
\usepackage{cite} % поддержка цитирования
\usepackage{hyperref} % создание гиперссылок

% Прочее
\usepackage{color} % работа с цветом
\usepackage{epstopdf} % конвертация eps в pdf
\usepackage{multirow} % объединение ячеек таблиц по вертикали
\usepackage{afterpage} % вставка материала после текущей страницы
\usepackage[font={normal}]{caption} % настройка подписей к рисункам и таблицам
\usepackage[onehalfspacing]{setspace} % полуторный интервал
\usepackage{fancyhdr} % установка колонтитулов
\usepackage{listings} % поддержка вставки исходного кода

% Установка шрифта Times New Roman
\renewcommand{\rmdefault}{ftm}

% Создание нового типа столбца для выравнивания содержимого по центру
\newcommand{\PreserveBackslash}[1]{\let\temp=\\#1\let\\=\temp}
\newcolumntype{C}[1]{>{\PreserveBackslash\centering}p{#1}}

% Настройка стиля страницы
\pagestyle{fancy}      % Использование стиля "fancy" для оформления страниц
\fancyhf{}              % Очистка текущих значений колонтитулов
\fancyfoot[C]{\thepage} % Установка номера страницы в нижнем колонтитуле по центру
\renewcommand{\headrulewidth}{0pt} % Удаление разделительной линии в верхнем колонтитуле

% Настройка подписей к изображениям и таблицам
\captionsetup{format=hang,labelsep=period}

% Использование полужирного начертания для векторов
\let\vec=\mathbf

% Установка глубины оглавления
\setcounter{tocdepth}{2}

% Указание папки для поиска изображений
\graphicspath{{images/}}

% Установка стилей страницы и главы
\pagestyle{footcenter}
\chapterpagestyle{footcenter}

\usepackage{ragged2e}
\usepackage{csquotes} % опционально, для лучшей типографики
\usepackage{microtype} % улучшает переносы и выравнивание
\usepackage{enumitem}
\usepackage{float}

\setlist{itemsep=0pt, parsep=0pt, topsep=0pt, partopsep=0pt}
\setstretch{1.5}
\setcounter{secnumdepth}{3}
\sloppy
\justifying
\usepackage{chngcntr}
\counterwithin{figure}{subsection}

\newcommand{\normal}{\fontsize{14pt}{21pt}\selectfont}

\usepackage{etoolbox}

% Нумерация с 1
\setcounter{section}{0}
\setcounter{subsection}{0}
\setcounter{subsubsection}{0}

% Кастомные заголовки
\newcommand{\customsection}[1]{%
  \stepcounter{section}%
  \setcounter{subsection}{0}%
  \setcounter{subsubsection}{0}%
  \addcontentsline{toc}{section}{\thesection\quad#1}%
  {\setstretch{1.5}%
   \noindent\hspace{1.25cm}\textbf{\fontsize{16pt}{21pt}\selectfont \thesection\quad#1}%
   \par}%
}

\newcommand{\customsubsection}[1]{%
  \stepcounter{subsection}%
  \setcounter{subsubsection}{0}%
  \addcontentsline{toc}{subsection}{\thesubsection\quad#1}%
  {\setstretch{1.5}%
   \noindent\hspace{1.25cm}\textbf{\fontsize{14pt}{19pt}\selectfont \thesubsection\quad#1}%
   \par}%
}

\newcommand{\customsubsubsection}[1]{%
  \stepcounter{subsubsection}%
  \addcontentsline{toc}{subsubsection}{\thesubsubsection\quad#1}%
  {\setstretch{1.5}%
   \noindent\hspace{1.25cm}\textbf{\fontsize{14pt}{19pt}\selectfont \thesubsubsection\quad#1}%
   \par}%
}
