\documentclass[
bachelor, % document type
subf, % use and configure subfig package for nested figure numbering
times % use Times font as main
]{disser}

% Кодировка и язык
\usepackage[T2A]{fontenc} % поддержка кириллицы
\usepackage[utf8]{inputenc} % кодировка исходного текста
\usepackage[english,russian]{babel} % переключение языков
\usepackage{listings}

\lstset{
  inputencoding=utf8,
  extendedchars=\true,
  basicstyle=\ttfamily,
  literate=
    {Ö}{{\"O}}1
    {Ä}{{\"A}}1
    {Ü}{{\"U}}1
    {ß}{{\ss}}1
    {ü}{{\"u}}1
    {ä}{{\"a}}1
    {ö}{{\"o}}1
    {~}{{\textasciitilde}}1
    {а}{{\selectfont\char224}}1
    {б}{{\selectfont\char225}}1
    {в}{{\selectfont\char226}}1
    {г}{{\selectfont\char227}}1
    {д}{{\selectfont\char228}}1
    {е}{{\selectfont\char229}}1
    {ё}{{\"e}}1
    {ж}{{\selectfont\char230}}1
    {з}{{\selectfont\char231}}1
    {и}{{\selectfont\char232}}1
    {й}{{\selectfont\char233}}1
    {к}{{\selectfont\char234}}1
    {л}{{\selectfont\char235}}1
    {м}{{\selectfont\char236}}1
    {н}{{\selectfont\char237}}1
    {о}{{\selectfont\char238}}1
    {п}{{\selectfont\char239}}1
    {р}{{\selectfont\char240}}1
    {с}{{\selectfont\char241}}1
    {т}{{\selectfont\char242}}1
    {у}{{\selectfont\char243}}1
    {ф}{{\selectfont\char244}}1
    {х}{{\selectfont\char245}}1
    {ц}{{\selectfont\char246}}1
    {ч}{{\selectfont\char247}}1
    {ш}{{\selectfont\char248}}1
    {щ}{{\selectfont\char249}}1
    {ъ}{{\selectfont\char250}}1
    {ы}{{\selectfont\char251}}1
    {ь}{{\selectfont\char252}}1
    {э}{{\selectfont\char253}}1
    {ю}{{\selectfont\char254}}1
    {я}{{\selectfont\char255}}1
    {А}{{\selectfont\char192}}1
    {Б}{{\selectfont\char193}}1
    {В}{{\selectfont\char194}}1
    {Г}{{\selectfont\char195}}1
    {Д}{{\selectfont\char196}}1
    {Е}{{\selectfont\char197}}1
    {Ё}{{\"E}}1
    {Ж}{{\selectfont\char198}}1
    {З}{{\selectfont\char199}}1
    {И}{{\selectfont\char200}}1
    {Й}{{\selectfont\char201}}1
    {К}{{\selectfont\char202}}1
    {Л}{{\selectfont\char203}}1
    {М}{{\selectfont\char204}}1
    {Н}{{\selectfont\char205}}1
    {О}{{\selectfont\char206}}1
    {П}{{\selectfont\char207}}1
    {Р}{{\selectfont\char208}}1
    {С}{{\selectfont\char209}}1
    {Т}{{\selectfont\char210}}1
    {У}{{\selectfont\char211}}1
    {Ф}{{\selectfont\char212}}1
    {Х}{{\selectfont\char213}}1
    {Ц}{{\selectfont\char214}}1
    {Ч}{{\selectfont\char215}}1
    {Ш}{{\selectfont\char216}}1
    {Щ}{{\selectfont\char217}}1
    {Ъ}{{\selectfont\char218}}1
    {Ы}{{\selectfont\char219}}1
    {Ь}{{\selectfont\char220}}1
    {Э}{{\selectfont\char221}}1
    {Ю}{{\selectfont\char222}}1
    {Я}{{\selectfont\char223}}1
}

% Геометрия страницы и графика
\usepackage[left=3cm, right=1.5cm, top=2cm, bottom=2cm]{geometry} % поля страницы
\usepackage{graphicx} % подключение графики
\usepackage{pdfpages} % вставка pdf-страниц

% Таблицы
\usepackage{array} % расширенные возможности для работы с таблицами
\usepackage{tabularx} % автоматический подбор ширины столбцов
\usepackage{dcolumn} % выравнивание чисел по разделителю

% Библиография и ссылки
\usepackage{cite} % поддержка цитирования
\usepackage{hyperref} % создание гиперссылок

% Прочее
\usepackage{color} % работа с цветом
\usepackage{epstopdf} % конвертация eps в pdf
\usepackage{multirow} % объединение ячеек таблиц по вертикали
\usepackage{afterpage} % вставка материала после текущей страницы
\usepackage[font={normal}]{caption} % настройка подписей к рисункам и таблицам
\usepackage[onehalfspacing]{setspace} % полуторный интервал
\usepackage{fancyhdr} % установка колонтитулов
\usepackage{capt-of}
\usepackage{xparse}

\captionsetup[figure]{
  position=bottom,
  justification=centering,
  singlelinecheck=true
}

\captionsetup[table]{
  position=top,
  justification=raggedright,
  singlelinecheck=false
}

% Тип caption для листинга
\DeclareCaptionType{listing}[Листинг][Список листингов]
\counterwithin{listing}{subsection}

\captionsetup[listing]{
  position=top,
  justification=raggedright,
  singlelinecheck=false,
}

\NewDocumentCommand{\codeblockfile}{O{} m m m}{%
  \refstepcounter{listing}
  \bigskip
  \noindent\begin{minipage}{\textwidth}
    {\raggedright
    Листинг \thelisting. #2\label{#3}
    \par\medskip}
    \lstinputlisting[#1]{#4}
  \end{minipage}
  \par\bigskip
}

\NewDocumentEnvironment{codeblockenv}{m m}
{
  \bigskip
  \noindent\begingroup
    \captionof{listing}{#1}\label{#2}
  \endgroup
  \par\bigskip
}
{
  \par
}

% Установка шрифта Times New Roman
\renewcommand{\rmdefault}{ftm}

% Создание нового типа столбца для выравнивания содержимого по центру
\newcommand{\PreserveBackslash}[1]{\let\temp=\\#1\let\\=\temp}
\newcolumntype{C}[1]{>{\PreserveBackslash\centering}p{#1}}

% Настройка стиля страницы
\pagestyle{fancy}      % Использование стиля "fancy" для оформления страниц
\fancyhf{}              % Очистка текущих значений колонтитулов
\fancyfoot[C]{\thepage} % Установка номера страницы в нижнем колонтитуле по центру
\renewcommand{\headrulewidth}{0pt} % Удаление разделительной линии в верхнем колонтитуле

% Установка глубины оглавления
\setcounter{tocdepth}{2}

% Указание папки для поиска изображений
\graphicspath{{images/}}

% Установка стилей страницы и главы
\pagestyle{footcenter}
\chapterpagestyle{footcenter}

\usepackage{ragged2e}
\usepackage{csquotes}
\usepackage{microtype}
\usepackage{enumitem}
\usepackage{float}
\usepackage{etoolbox}
\usepackage{chngcntr}

\setlist{itemsep=0pt, parsep=0pt, topsep=0pt, partopsep=0pt}
\setstretch{1.5}
\setcounter{secnumdepth}{3}
\sloppy
\justifying
\counterwithin{figure}{subsection}
\counterwithin{table}{subsection}
\setcounter{section}{0}
\setcounter{subsection}{0}
\setcounter{subsubsection}{0}

\newcommand{\normal}{\fontsize{14pt}{21pt}\selectfont}

% Кастомные заголовки
\newcommand{\customsection}[1]{
  \refstepcounter{section}
  \setcounter{subsection}{0}
  \setcounter{subsubsection}{0}
  \addcontentsline{toc}{section}{\thesection\quad#1}
  \phantomsection
  {\setstretch{1.5}
   \noindent\hspace{1.25cm}\textbf{\fontsize{16pt}{21pt}\selectfont \thesection\quad#1}
   \par}
}

\newcommand{\customsubsection}[1]{
  \refstepcounter{subsection}
  \setcounter{subsubsection}{0}
  \addcontentsline{toc}{subsection}{\thesubsection\quad#1}
  \phantomsection
  {\setstretch{1.5}
   \noindent\hspace{1.25cm}\textbf{\fontsize{14pt}{19pt}\selectfont \thesubsection\quad#1}
   \par}
}

\newcommand{\customsubsubsection}[1]{
  \refstepcounter{subsubsection}
  \addcontentsline{toc}{subsubsection}{\thesubsubsection\quad#1}
  \phantomsection
  {\setstretch{1.5}
   \noindent\hspace{1.25cm}\textbf{\fontsize{14pt}{19pt}\selectfont \thesubsubsection\quad#1}
   \par}
}