\hypertarget{ux442ux438ux442ux443ux43b-todo}{%
\section{\texorpdfstring{\textbf{Титул
ToDo:}}{Титул ToDo:}}\label{ux442ux438ux442ux443ux43b-todo}}

Разработка и внедрение системы мониторинга новостных ресурсов:
реализация фильтрационного слоя, развертывание микросервисной
архитектуры и обеспечение отказоустойчивости компонентов

\hypertarget{section}{%
\subsection{}\label{section}}

\hypertarget{ux440ux435ux444ux435ux440ux430ux442}{%
\section{Реферат}\label{ux440ux435ux444ux435ux440ux430ux442}}

\hypertarget{ux43eux431ux44aux451ux43c-ux440ux430ux431ux43eux442ux44b}{%
\subsection{Объём
работы:}\label{ux43eux431ux44aux451ux43c-ux440ux430ux431ux43eux442ux44b}}

Выпускная квалификационная работа изложена на \textbf{54 страницах},
содержит \textbf{14 рисунков}, \textbf{4 таблицы}, \textbf{30
источников}, \textbf{3 приложения}.

\hypertarget{ux43aux43bux44eux447ux435ux432ux44bux435-ux441ux43bux43eux432ux430}{%
\subsection{Ключевые
слова:}\label{ux43aux43bux44eux447ux435ux432ux44bux435-ux441ux43bux43eux432ux430}}

фильтрационный слой, мониторинг новостей, дедупликация, ключевые слова,
анализ тональности, отказоустойчивость, Docker, ElasticSearch, RabbitMQ,
MongoDB репликация, мониторинг компонентов, логирование.

\hypertarget{ux442ux435ux43aux441ux442-ux440ux435ux444ux435ux440ux430ux442ux430}{%
\subsection{Текст
реферата:}\label{ux442ux435ux43aux441ux442-ux440ux435ux444ux435ux440ux430ux442ux430}}

Выпускная квалификационная работа посвящена разработке и внедрению
системы мониторинга новостных ресурсов с акцентом на реализацию
фильтрационного слоя и обеспечению отказоустойчивости компонентов
системы.\\
В работе разработан фильтрационный слой, включающий функции дедупликации
новостных сообщений, фильтрации по ключевым словам и анализа тональности
публикаций.\\
Рассмотрены и реализованы механизмы развертывания компонентов системы в
контейнерной среде Docker, с использованием MongoDB в режиме репликации,
RabbitMQ и ElasticSearch.\\
Предложены методы обеспечения отказоустойчивости и мониторинга
компонентов, включая автоматическое восстановление сервисов и систему
логирования событий.\\
Результаты работы демонстрируют повышение надёжности и
отказоустойчивости системы мониторинга новостей, а также улучшение
качества фильтрации данных. Разработанная система может быть
использована для мониторинга репутации в СМИ и социальных сетях, а также
для информационных служб компаний и государственных структур.

\hypertarget{ux43eux431ux43eux437ux43dux430ux447ux435ux43dux438ux44f-ux438-ux441ux43eux43aux440ux430ux449ux435ux43dux438ux44f}{%
\section{Обозначения и
сокращения}\label{ux43eux431ux43eux437ux43dux430ux447ux435ux43dux438ux44f-ux438-ux441ux43eux43aux440ux430ux449ux435ux43dux438ux44f}}

\textbf{Фильтрационный слой} --- компонент системы, выполняющий отбор,
анализ и обработку новостных данных по заданным критериям.

\textbf{Дедупликация} --- процесс удаления повторяющихся данных или
сообщений из потока новостей.

\textbf{Ключевые слова} --- заранее определённый набор слов или фраз, по
которым осуществляется фильтрация новостных сообщений.

\textbf{Анализ тональности (Sentiment Analysis)} --- метод обработки
текстовой информации, позволяющий определить эмоциональную окраску
сообщения (положительную, отрицательную или нейтральную).

\textbf{RabbitMQ} --- система маршрутизации сообщений, обеспечивающая
асинхронное взаимодействие микросервисов и обработку событий.

\textbf{MongoDB} --- база данных, применяемая для хранения
структурированных и полуструктурированных новостных данных, а также
метаинформации о фильтрации.

\textbf{ElasticSearch} --- поисковый движок, используемый для быстрого
индексирования и поиска информации в больших массивах новостей.

\textbf{Docker} --- среда контейнеризации, применяемая для развёртывания
компонентов системы в изолированных средах с возможностью
масштабирования.

\textbf{Репликация MongoDB} --- механизм повышения отказоустойчивости и
доступности данных за счёт создания нескольких копий базы данных.

\textbf{Отказоустойчивость} --- способность системы продолжать
функционирование в случае отказа отдельных её компонентов.

\textbf{Мониторинг состояния компонентов} --- процесс отслеживания
работоспособности сервисов и систем с целью своевременного обнаружения
сбоев и отклонений.

\textbf{Логирование} --- процесс регистрации событий, происходящих в
системе, с целью последующего анализа и диагностики.

\hypertarget{ux432ux432ux435ux434ux435ux43dux438ux435}{%
\section{Введение}\label{ux432ux432ux435ux434ux435ux43dux438ux435}}

\hypertarget{ux430ux43aux442ux443ux430ux43bux44cux43dux43eux441ux442ux44c-ux442ux435ux43cux44b}{%
\subsection{Актуальность
темы}\label{ux430ux43aux442ux443ux430ux43bux44cux43dux43eux441ux442ux44c-ux442ux435ux43cux44b}}

В современном мире, характеризующемся высоким темпом появления и
распространения информации, особенно актуальными становятся задачи
мониторинга и анализа новостных ресурсов. Большие объёмы данных,
поступающих из различных источников (RSS-ленты, социальные сети,
мессенджеры), требуют использования автоматизированных систем обработки
и фильтрации контента. Одной из ключевых проблем является не только
своевременное получение данных, но и их качественная фильтрация по
критериям релевантности, достоверности и значимости для конечного
пользователя.

Повышенные требования к надёжности и стабильности таких систем
обусловливают необходимость использования отказоустойчивых решений и
развертывания компонентов в распределённой среде. Для этого применяются
микросервисные архитектуры с обеспечением мониторинга и высокой
доступности всех сервисов.

\hypertarget{ux446ux435ux43bux44c-ux438-ux437ux430ux434ux430ux447ux438-ux438ux441ux441ux43bux435ux434ux43eux432ux430ux43dux438ux44f}{%
\subsection{Цель и задачи
исследования}\label{ux446ux435ux43bux44c-ux438-ux437ux430ux434ux430ux447ux438-ux438ux441ux441ux43bux435ux434ux43eux432ux430ux43dux438ux44f}}

Целью данной работы является разработка фильтрационного слоя системы
мониторинга новостных ресурсов и внедрение микросервисной архитектуры с
обеспечением отказоустойчивости её компонентов.

В рамках достижения цели решаются следующие задачи:

\begin{enumerate}
\def\labelenumi{\arabic{enumi}.}
\tightlist
\item
  Реализация фильтрационного слоя с поддержкой функций:

  \begin{itemize}
  \tightlist
  \item
    дедупликации новостных сообщений;\\
  \item
    фильтрации по ключевым словам;\\
  \item
    определения тональности публикаций;\\
  \item
    определения потенциальной вирусности публикации;\\
  \end{itemize}
\item
  Развертывание компонентов системы на базе контейнерной платформы
  Docker.\\
\item
  Обеспечение отказоустойчивости системы на уровне микросервисов и
  инфраструктуры.\\
\item
  Настройка систем логирования.\\
\item
  Проведение тестирования отказоустойчивости системы.
\end{enumerate}

\hypertarget{ux43eux431ux44aux435ux43aux442-ux438-ux43fux440ux435ux434ux43cux435ux442-ux438ux441ux441ux43bux435ux434ux43eux432ux430ux43dux438ux44f}{%
\subsection{Объект и предмет
исследования}\label{ux43eux431ux44aux435ux43aux442-ux438-ux43fux440ux435ux434ux43cux435ux442-ux438ux441ux441ux43bux435ux434ux43eux432ux430ux43dux438ux44f}}

\textbf{Объект исследования} --- система мониторинга новостных ресурсов
и потоков данных из различных источников.\\
\textbf{Предмет исследования} --- архитектура отказоустойчивой системы
обработки данных с реализацией фильтрационного слоя.

\hypertarget{ux43cux435ux442ux43eux434ux44b-ux438ux441ux441ux43bux435ux434ux43eux432ux430ux43dux438ux44f}{%
\subsection{Методы
исследования}\label{ux43cux435ux442ux43eux434ux44b-ux438ux441ux441ux43bux435ux434ux43eux432ux430ux43dux438ux44f}}

В работе применяются методы построения микросервисных систем,
организации отказоустойчивых инфраструктурных решений, алгоритмы
обработки текстовых данных и анализа тональности, методы дедупликации и
выявления аномалий. Используются технологии Docker, MongoDB, RabbitMQ,
ElasticSearch и языки программирования Python и C\#.

\hypertarget{section-1}{%
\subsection{}\label{section-1}}

\hypertarget{ux43fux440ux430ux43aux442ux438ux447ux435ux441ux43aux430ux44f-ux437ux43dux430ux447ux438ux43cux43eux441ux442ux44c}{%
\subsection{Практическая
значимость}\label{ux43fux440ux430ux43aux442ux438ux447ux435ux441ux43aux430ux44f-ux437ux43dux430ux447ux438ux43cux43eux441ux442ux44c}}

Результатом исследования является внедрённая система мониторинга
новостей, обеспечивающая высокую отказоустойчивость и достоверность
фильтрации информации. Решение может быть применено в системах
корпоративной аналитики, службах мониторинга репутации и СМИ, а также в
государственных информационных системах.

\hypertarget{ux441ux442ux440ux443ux43aux442ux443ux440ux430-ux440ux430ux431ux43eux442ux44b}{%
\subsection{Структура
работы}\label{ux441ux442ux440ux443ux43aux442ux443ux440ux430-ux440ux430ux431ux43eux442ux44b}}

Работа состоит из введения, трёх глав, заключения, списка использованных
источников и приложений.

\begin{enumerate}
\def\labelenumi{\arabic{enumi}.}
\tightlist
\item
  В первой главе рассматриваются существующие подходы к мониторингу
  новостей и фильтрации данных, а также архитектурные решения для
  построения отказоустойчивых систем.\\
\item
  Во второй главе выполняется постановка задачи\\
\item
  Во третьей главе описывается реализация фильтрационного слоя
  новостей.\\
\item
  В четвертой главе приведены решения по развертыванию компонентов
  системы и обеспечению их отказоустойчивости.
\end{enumerate}

\hypertarget{ux430ux43dux430ux43bux438ux437-ux43cux435ux442ux43eux434ux43eux432-ux444ux438ux43bux44cux442ux440ux430ux446ux438ux438-ux43dux43eux432ux43eux441ux442ux43dux43eux433ux43e-ux43aux43eux43dux442ux435ux43dux442ux430-ux438-ux43fux43eux441ux442ux440ux43eux435ux43dux438ux44f-ux43eux442ux43aux430ux437ux43eux443ux441ux442ux43eux439ux447ux438ux432ux44bux445-ux441ux438ux441ux442ux435ux43c}{%
\section{1. Анализ методов фильтрации новостного контента и построения
отказоустойчивых
систем}\label{ux430ux43dux430ux43bux438ux437-ux43cux435ux442ux43eux434ux43eux432-ux444ux438ux43bux44cux442ux440ux430ux446ux438ux438-ux43dux43eux432ux43eux441ux442ux43dux43eux433ux43e-ux43aux43eux43dux442ux435ux43dux442ux430-ux438-ux43fux43eux441ux442ux440ux43eux435ux43dux438ux44f-ux43eux442ux43aux430ux437ux43eux443ux441ux442ux43eux439ux447ux438ux432ux44bux445-ux441ux438ux441ux442ux435ux43c}}

\hypertarget{ux430ux43dux430ux43bux438ux437-ux441ux443ux449ux435ux441ux442ux432ux443ux44eux449ux438ux445-ux440ux435ux448ux435ux43dux438ux439-ux43cux43eux43dux438ux442ux43eux440ux438ux43dux433ux430-ux43dux43eux432ux43eux441ux442ux43dux43eux433ux43e-ux43aux43eux43dux442ux435ux43dux442ux430}{%
\subsection{1.1 Анализ существующих решений мониторинга новостного
контента}\label{ux430ux43dux430ux43bux438ux437-ux441ux443ux449ux435ux441ux442ux432ux443ux44eux449ux438ux445-ux440ux435ux448ux435ux43dux438ux439-ux43cux43eux43dux438ux442ux43eux440ux438ux43dux433ux430-ux43dux43eux432ux43eux441ux442ux43dux43eux433ux43e-ux43aux43eux43dux442ux435ux43dux442ux430}}

\hypertarget{ux434ux435ux434ux443ux43fux43bux438ux43aux430ux446ux438ux44f}{%
\subsubsection{\texorpdfstring{\textbf{1.1.1
Дедупликация}}{1.1.1 Дедупликация}}\label{ux434ux435ux434ux443ux43fux43bux438ux43aux430ux446ux438ux44f}}

Дедупликация --- ключевой этап в обработке новостного контента,
направленный на устранение повторяющихся или схожих сообщений.
Существуют различные методы дедупликации:

\begin{itemize}
\tightlist
\item
  \textbf{Хеширование}: использование хеш-функций для идентификации
  идентичных текстов.\\
\item
  \textbf{Сравнение по ключевым словам}: выделение и сравнение ключевых
  слов в текстах для определения схожести.\\
\item
  \textbf{Семантический анализ}: применение методов обработки
  естественного языка для выявления смысловой близости текстов.
\end{itemize}

TODOСовременные исследования предлагают адаптивные методы дедупликации,
учитывающие контекст и структуру данных, что повышает точность
фильтрации.

\hypertarget{ux444ux438ux43bux44cux442ux440ux430ux446ux438ux44f-ux43fux43e-ux43aux43bux44eux447ux435ux432ux44bux43c-ux441ux43bux43eux432ux430ux43c}{%
\subsubsection{\texorpdfstring{\textbf{1.1.2 Фильтрация по ключевым
словам}}{1.1.2 Фильтрация по ключевым словам}}\label{ux444ux438ux43bux44cux442ux440ux430ux446ux438ux44f-ux43fux43e-ux43aux43bux44eux447ux435ux432ux44bux43c-ux441ux43bux43eux432ux430ux43c}}

Фильтрация по ключевым словам является основным инструментом для отбора
релевантного новостного контента. Существуют различные подходы:

\begin{itemize}
\tightlist
\item
  \textbf{Прямое сопоставление}: поиск точных совпадений ключевых слов в
  тексте.\\
\item
  \textbf{Использование регулярных выражений}: позволяет учитывать
  различные формы слов и фраз.\\
\item
  \textbf{Семантический анализ}: учет синонимов и контекста для более
  гибкой фильтрации.
\end{itemize}

Инструменты, такие как Elasticsearch, предоставляют мощные возможности
для реализации фильтрации по ключевым словам с использованием различных
методов анализа текста.

\hypertarget{ux43eux43fux440ux435ux434ux435ux43bux435ux43dux438ux435-ux44dux43cux43eux446ux438ux43eux43dux430ux43bux44cux43dux43eux433ux43e-ux43eux43aux440ux430ux441ux430}{%
\subsubsection{\texorpdfstring{\textbf{1.1.3 Определение эмоционального
окраса}}{1.1.3 Определение эмоционального окраса}}\label{ux43eux43fux440ux435ux434ux435ux43bux435ux43dux438ux435-ux44dux43cux43eux446ux438ux43eux43dux430ux43bux44cux43dux43eux433ux43e-ux43eux43aux440ux430ux441ux430}}

Анализ эмоционального окраса (сентимент-анализ) позволяет
классифицировать новостные сообщения по тональности: положительной,
отрицательной или нейтральной. Существуют различные методы:

\begin{itemize}
\tightlist
\item
  \textbf{Словарные методы}: использование заранее составленных словарей
  с оценкой эмоциональной нагрузки слов.\\
\item
  \textbf{Машинное обучение}: обучение моделей на размеченных данных для
  определения тональности.\\
\item
  \textbf{Гибридные подходы}: сочетание словарных методов и машинного
  обучения для повышения точности.
\end{itemize}

Современные исследования подчеркивают эффективность гибридных методов,
особенно при анализе коротких текстов, таких как заголовки новостей.

\hypertarget{ux432ux44bux447ux438ux441ux43bux435ux43dux438ux435-ux43fux43eux442ux435ux43dux446ux438ux430ux43bux44cux43dux43eux439-ux432ux438ux440ux443ux441ux43dux43eux441ux442ux438}{%
\subsubsection{\texorpdfstring{\textbf{1.1.4 Вычисление потенциальной
вирусности}}{1.1.4 Вычисление потенциальной вирусности}}\label{ux432ux44bux447ux438ux441ux43bux435ux43dux438ux435-ux43fux43eux442ux435ux43dux446ux438ux430ux43bux44cux43dux43eux439-ux432ux438ux440ux443ux441ux43dux43eux441ux442ux438}}

Оценка потенциальной вирусности новостного контента представляет собой
задачу предсказания вероятности быстрого и широкого распространения
материала в информационной среде. Для этого используются методы из
областей машинного обучения, анализа графов и анализа текстов.

Наиболее распространённые подходы включают:

\begin{itemize}
\item
  \textbf{Регрессионные модели (Logistic Regression, Random Forest)}\\
  Используются для предсказания бинарного события --- станет ли материал
  вирусным или нет, на основе метаданных (время публикации, количество
  слов, источник, первые реакции).\\
  \textbf{Источник:} Ma, J., Gao, W., \& Wong, K.-F. (2018). Rumor
  detection on Twitter with tree-structured recursive neural networks.
  \emph{ACL 2018}.\\
\item
  \textbf{Глубокие нейронные сети (CNN, RNN, Transformer)}\\
  Модели, такие как BERT или LSTM, обучаются на текстах новостей с
  учетом их предыдущей популярности (например, число репостов или
  лайков). Это позволяет учитывать не только содержание, но и контекст
  публикации.\\
  \textbf{Источник:} Nguyen, D. T., Sugiyama, K., Nakov, P., \& Kan,
  M.-Y. (2020). FANG: Leveraging social context for fake news detection
  using graph representation. \emph{CIKM}.\\
\item
  \textbf{Графовые модели распространения (Graph Propagation Models)}\\
  Контент рассматривается как узел в сети, а пользователи и связи между
  ними --- как ребра. Применяются модели, такие как Independent Cascade
  или Linear Threshold для симуляции распространения.\\
  \textbf{Источник:} Kempe, D., Kleinberg, J., \& Tardos, É. (2003).
  Maximizing the spread of influence through a social network.
  \emph{KDD}.
\item
  \textbf{Векторизация контента с обучением на исторических данных
  (TF-IDF, Doc2Vec, BERT embeddings)}\\
  Содержимое новости преобразуется в вектор признаков, и на его основе
  оценивается схожесть с уже известными вирусными публикациями.\\
  \textbf{Источник:} Bandari, R., Asur, S., \& Huberman, B. A. (2012).
  The Pulse of News in Social Media: Forecasting Popularity.
  \emph{ICWSM}.
\item
  \textbf{Гибридные модели (мультимодальные)}\\
  Используются одновременно текст, метаданные и поведенческие данные
  (время просмотра, вовлечённость). Такие модели комбинируют разные типы
  входных признаков и обычно реализуются на основе ансамблей или
  нейросетей с несколькими входами.\\
  \textbf{Источник:} Tatar, A., Antoniadis, P., De Amorim, M. D., \&
  Fdida, S. (2014). A Survey on Predicting the Popularity of Web
  Content. \emph{Computer Communications}, 36(11-12), 1132-1144.
\end{itemize}

Таким образом, вычисление вирусности представляет собой комплексную
задачу, в которой используются как традиционные алгоритмы машинного
обучения, так и современные методы анализа графов и нейросетевые
архитектуры. Выбор подхода зависит от доступных данных и требуемой
точности модели.

\hypertarget{ux430ux43dux430ux43bux438ux437-ux441ux443ux449ux435ux441ux442ux432ux443ux44eux449ux438ux445-ux43fux43eux434ux445ux43eux434ux43eux432-ux43a-ux43fux43eux441ux442ux440ux43eux435ux43dux438ux44e-ux43eux442ux43aux430ux437ux43eux443ux441ux442ux43eux439ux447ux438ux432ux44bux445-ux441ux438ux441ux442ux435ux43c}{%
\subsection{\texorpdfstring{\textbf{1.2 Анализ существующих подходов к
построению отказоустойчивых
систем}}{1.2 Анализ существующих подходов к построению отказоустойчивых систем}}\label{ux430ux43dux430ux43bux438ux437-ux441ux443ux449ux435ux441ux442ux432ux443ux44eux449ux438ux445-ux43fux43eux434ux445ux43eux434ux43eux432-ux43a-ux43fux43eux441ux442ux440ux43eux435ux43dux438ux44e-ux43eux442ux43aux430ux437ux43eux443ux441ux442ux43eux439ux447ux438ux432ux44bux445-ux441ux438ux441ux442ux435ux43c}}

Отказоустойчивость --- способность системы продолжать функционировать
при возникновении сбоев. Современные подходы к построению
отказоустойчивых систем включают:

\begin{itemize}
\tightlist
\item
  \textbf{Микросервисная архитектура}: разделение системы на независимые
  сервисы, что позволяет локализовать сбои и облегчает
  масштабирование.\\
\item
  \textbf{Использование оркестраторов}: инструменты, такие как Docker
  Swarm, управляют развертыванием, масштабированием и восстановлением
  сервисов.\\
\item
  \textbf{Паттерны отказоустойчивости}: применение шаблонов
  проектирования, таких как Circuit Breaker и Retry, для обработки
  сбоев.
\end{itemize}

\hypertarget{ux430ux43dux430ux43bux438ux437-ux441ux443ux449ux435ux441ux442ux432ux443ux44eux449ux438ux445-ux440ux435ux448ux435ux43dux438ux439-ux441ux431ux43eux440ux430-ux442ux435ux43bux435ux43cux435ux442ux440ux438ux438-ux441ux438ux441ux442ux435ux43cux44b-ux438-ux445ux440ux430ux43dux435ux43dux438ux44f-ux43bux43eux433ux43eux432}{%
\subsection{\texorpdfstring{\textbf{1.3 Анализ существующих решений
сбора телеметрии системы и хранения
логов}}{1.3 Анализ существующих решений сбора телеметрии системы и хранения логов}}\label{ux430ux43dux430ux43bux438ux437-ux441ux443ux449ux435ux441ux442ux432ux443ux44eux449ux438ux445-ux440ux435ux448ux435ux43dux438ux439-ux441ux431ux43eux440ux430-ux442ux435ux43bux435ux43cux435ux442ux440ux438ux438-ux441ux438ux441ux442ux435ux43cux44b-ux438-ux445ux440ux430ux43dux435ux43dux438ux44f-ux43bux43eux433ux43eux432}}

Эффективный сбор телеметрии и логов необходим для мониторинга и
диагностики систем. Современные решения включают:

\begin{itemize}
\tightlist
\item
  \textbf{Системы сбора метрик}: инструменты, такие как Prometheus,
  собирают и хранят метрики производительности.\\
\item
  \textbf{Системы агрегации логов}: инструменты, такие как Filebeat,
  собирают, обрабатывают и передают логи в хранилища.\\
\item
  \textbf{Платформы визуализации}: инструменты, такие как Kibana,
  предоставляют визуальное представление метрик и логов для анализа.
\end{itemize}

Интеграция этих инструментов обеспечивает полную наблюдаемость системы,
позволяя оперативно выявлять и устранять проблемы.

\hypertarget{ux432ux44bux432ux43eux434ux44b-ux43fux43e-ux433ux43bux430ux432ux435-1}{%
\subsection{1.7. Выводы по главе
1}\label{ux432ux44bux432ux43eux434ux44b-ux43fux43e-ux433ux43bux430ux432ux435-1}}

В ходе анализа существующих решений в области мониторинга новостного
контента и построения отказоустойчивых систем были сделаны следующие
выводы:

\begin{enumerate}
\def\labelenumi{\arabic{enumi}.}
\tightlist
\item
  \textbf{Методы фильтрации контента} демонстрируют значительное
  разнообразие как по глубине анализа, так и по вычислительным затратам.
  Простейшие подходы, такие как хеширование и фильтрация по ключевым
  словам, обеспечивают высокую производительность, но ограничены в
  выявлении смысловых дубликатов или релевантных публикаций в изменённой
  формулировке. Более продвинутые методы --- семантический анализ и
  гибридные модели --- требуют больших вычислительных ресурсов, но дают
  более точные результаты и способны учитывать контекст.\\
\item
  \textbf{Определение эмоционального окраса и оценка вирусности}
  основаны на широком спектре методов: от словарных и регрессионных до
  графовых и нейросетевых. Эффективное использование этих методов
  позволяет не только фильтровать контент по тону, но и прогнозировать
  его дальнейшее распространение в сети, что критично для новостного
  мониторинга в высоконагруженных информационных средах.\\
\item
  \textbf{Современные подходы к построению отказоустойчивых систем},
  такие как микросервисная архитектура в сочетании с Docker Swarm,
  обеспечивают гибкость масштабирования и изоляцию сбоев. Использование
  паттернов обработки ошибок, включая политику повторных попыток
  (retry), позволяет значительно повысить надёжность работы сервисов.\\
\item
  \textbf{Системы наблюдаемости и логирования}, основанные на связке
  Filebeat, Elasticsearch и Kibana, являются промышленным стандартом для
  обеспечения прозрачности работы компонентов системы. Они позволяют
  своевременно выявлять отклонения в поведении сервисов и проводить
  диагностику на основе собранных логов и метрик.
\end{enumerate}

Таким образом, анализ показал, что эффективная система мониторинга
новостного контента должна сочетать в себе современные методы
фильтрации, механизмы предсказания распространения информации,
архитектурные решения для отказоустойчивости и развитую систему
телеметрии. Эти аспекты легли в основу проектирования и реализации
предлагаемого решения, рассмотренного в следующих главах работы.

\hypertarget{ux43fux43eux441ux442ux430ux43dux43eux432ux43aux430-ux437ux430ux434ux430ux447ux438}{%
\section{2. Постановка
задачи}\label{ux43fux43eux441ux442ux430ux43dux43eux432ux43aux430-ux437ux430ux434ux430ux447ux438}}

\hypertarget{ux43eux431ux449ux430ux44f-ux445ux430ux440ux430ux43aux442ux435ux440ux438ux441ux442ux438ux43aux430-ux440ux435ux448ux430ux435ux43cux43eux439-ux437ux430ux434ux430ux447ux438}{%
\subsection{2.1. Общая характеристика решаемой
задачи}\label{ux43eux431ux449ux430ux44f-ux445ux430ux440ux430ux43aux442ux435ux440ux438ux441ux442ux438ux43aux430-ux440ux435ux448ux430ux435ux43cux43eux439-ux437ux430ux434ux430ux447ux438}}

Современные информационные системы ежедневно генерируют огромные объёмы
новостных данных. Одной из ключевых задач является не только оперативный
сбор такой информации, но и её эффективная фильтрация с целью
предоставления конечным пользователям только актуального, достоверного и
релевантного контента.

Решение данной задачи требует реализации высокоэффективных алгоритмов
фильтрации и анализа тональности сообщений. Дополнительно возникает
необходимость в обеспечении отказоустойчивости всей системы для её
непрерывного функционирования в условиях возможных сбоев отдельных
компонентов.

\hypertarget{ux446ux435ux43bux44c-ux440ux430ux431ux43eux442ux44b}{%
\subsection{2.2. Цель
работы}\label{ux446ux435ux43bux44c-ux440ux430ux431ux43eux442ux44b}}

Целью настоящей работы является разработка фильтрационного слоя системы
мониторинга новостных ресурсов, а также внедрение архитектурных решений,
обеспечивающих отказоустойчивую работу компонентов системы.

\hypertarget{ux437ux430ux434ux430ux447ux438-ux438ux441ux441ux43bux435ux434ux43eux432ux430ux43dux438ux44f}{%
\subsection{2.4. Задачи
исследования}\label{ux437ux430ux434ux430ux447ux438-ux438ux441ux441ux43bux435ux434ux43eux432ux430ux43dux438ux44f}}

В рамках работы решаются следующие задачи:

\begin{enumerate}
\def\labelenumi{\arabic{enumi}.}
\tightlist
\item
  Разработать фильтрационный слой, обеспечивающий:

  \begin{itemize}
  \tightlist
  \item
    удаление дублирующихся сообщений (дедупликация);\\
  \item
    фильтрацию контента по ключевым словам;\\
  \item
    определение тональности публикаций (положительная, отрицательная,
    нейтральная);\\
  \item
    определение потенциальной вирусности публикации;\\
  \end{itemize}
\item
  Развернуть инфраструктуру системы мониторинга новостей, включающую:

  \begin{itemize}
  \tightlist
  \item
    MongoDB в режиме репликации для повышения доступности данных;\\
  \item
    RabbitMQ в кластерной конфигурации для обеспечения надёжной передачи
    сообщений между сервисами;\\
  \item
    ElasticSearch для обеспечения поиска и аналитики новостных данных.\\
  \item
    Сервисов парсинга, обработки данных, API и Telegram-бот.\\
  \end{itemize}
\item
  Организовать развёртывание сервисов в контейнерной среде Docker для
  упрощения масштабирования и управления жизненным циклом компонентов.\\
\item
  Реализовать механизмы мониторинга и логирования компонентов системы с
  целью обеспечения их отказоустойчивости и быстрого обнаружения
  сбоев.\\
\item
  Провести тестирование системы на предмет надёжности и стабильности
  функционирования при различных нагрузках.
\end{enumerate}

\hypertarget{ux43eux436ux438ux434ux430ux435ux43cux44bux435-ux440ux435ux437ux443ux43bux44cux442ux430ux442ux44b}{%
\subsection{2.5. Ожидаемые
результаты}\label{ux43eux436ux438ux434ux430ux435ux43cux44bux435-ux440ux435ux437ux443ux43bux44cux442ux430ux442ux44b}}

Ожидается, что результатом работы станет развернутая система мониторинга
новостных ресурсов с реализованным фильтрационным слоем и архитектурой,
обеспечивающей отказоустойчивость её основных компонентов.

Система должна демонстрировать высокую устойчивость к сбоям,
обеспечивать фильтрацию данных в реальном времени и предоставлять
пользователям только актуальную и релевантную информацию.

\hypertarget{ux440ux435ux430ux43bux438ux437ux430ux446ux438ux44f-ux444ux438ux43bux44cux442ux440ux430ux446ux438ux43eux43dux43dux43eux433ux43e-ux441ux43bux43eux44f-ux43dux43eux432ux43eux441ux442ux435ux439}{%
\section{3. Реализация фильтрационного слоя
новостей}\label{ux440ux435ux430ux43bux438ux437ux430ux446ux438ux44f-ux444ux438ux43bux44cux442ux440ux430ux446ux438ux43eux43dux43dux43eux433ux43e-ux441ux43bux43eux44f-ux43dux43eux432ux43eux441ux442ux435ux439}}

\hypertarget{ux43eux431ux43eux441ux43dux43eux432ux430ux43dux438ux435-ux432ux44bux431ux43eux440ux430-ux442ux435ux445ux43dux43eux43bux43eux433ux438ux439}{%
\subsection{3.1. Обоснование выбора
технологий}\label{ux43eux431ux43eux441ux43dux43eux432ux430ux43dux438ux435-ux432ux44bux431ux43eux440ux430-ux442ux435ux445ux43dux43eux43bux43eux433ux438ux439}}

Выбор технологий для реализации системы мониторинга новостных ресурсов
и, в частности, фильтрационного слоя, базируется на результатах анализа,
представленного в главе 1. В рамках анализа были выделены ключевые
требования к системе: поддержка масштабируемой микросервисной
архитектуры, обеспечение отказоустойчивости, высокая наблюдаемость, а
также применение современных методов фильтрации контента, включая
дедупликацию, анализ тональности и оценку потенциальной вирусности.

\begin{itemize}
\tightlist
\item
  \textbf{Микросервисный подход} в совместной работе был выбран как
  базовая архитектурная парадигма ввиду его соответствия принципам
  отказоустойчивости и масштабируемости (см. п. 1.2). Он позволяет
  изолировать критически важные компоненты, упростить развертывание и
  ускорить развитие системы.\\
\item
  В качестве \textbf{инструмента оркестрации контейнеров} выбран
  \textbf{Docker Compose} с возможностью миграции на \textbf{Docker
  Swarm}. Это решение обеспечивает автоматизированное развертывание и
  поддержку отказоустойчивых сценариев при сбоях отдельных сервисов.\\
\item
  Для \textbf{фильтрации контента} реализован ряд механизмов, описанных
  в разделе 1.1: дедупликация на основе семантического сходства (см.
  1.1.1), фильтрация по ключевым словам (см. 1.1.2), анализ тональности
  с использованием предобученной модели для русского языка (см. 1.1.3),
  а также вычисление вирусности с применением batched inference через
  REST API (см. 1.1.4).\\
\item
  Для \textbf{сбора логов и телеметрии} выбрано решение на базе
  \textbf{Filebeat + Elasticsearch + Kibana} (см. 1.3), обеспечивающее
  полную наблюдаемость за всеми компонентами системы, включая анализ
  ошибок, метрик загрузки и отслеживание состояния очередей.\\
\item
  Во всех синхронных взаимодействиях между микросервисами реализована
  \textbf{политика повторных попыток (retry policy)}, что соответствует
  рекомендациям по построению отказоустойчивых распределённых систем и
  снижает вероятность потери данных при кратковременных сбоях (см. п.
  1.2).
\end{itemize}

Таким образом, выбранный стек технологий и архитектурные решения
напрямую опираются на выявленные в исследовании требования к системе, а
также соответствуют современным промышленным практикам построения
отказоустойчивых и интеллектуальных систем обработки данных.

\hypertarget{ux43eux431ux449ux430ux44f-ux430ux440ux445ux438ux442ux435ux43aux442ux443ux440ux430-ux441ux438ux441ux442ux435ux43cux44b-ux43cux43eux43dux438ux442ux43eux440ux438ux43dux433ux430-ux43dux43eux432ux43eux441ux442ux435ux439}{%
\subsection{3.2 Общая архитектура системы мониторинга
новостей}\label{ux43eux431ux449ux430ux44f-ux430ux440ux445ux438ux442ux435ux43aux442ux443ux440ux430-ux441ux438ux441ux442ux435ux43cux44b-ux43cux43eux43dux438ux442ux43eux440ux438ux43dux433ux430-ux43dux43eux432ux43eux441ux442ux435ux439}}

Разрабатываемая система мониторинга новостей построена на основе
микросервисной архитектуры, в которой каждый компонент отвечает за
строго определённую функцию в процессе обработки новостного потока.
Основу взаимодействия между компонентами составляет асинхронная
коммуникация через брокер сообщений \textbf{RabbitMQ}, а для хранения
данных используется \textbf{MongoDB} и \textbf{Elasticsearch}.

На входе системы работают демоны-парсеры, ответственные за сбор
новостной информации из различных источников: \textbf{социальных сетей}
(VK, Telegram) и \textbf{RSS-лент}. Каждый демон осуществляет
непрерывный мониторинг доверенных источников, извлекает новые публикации
и помещает их в очередь сообщений \texttt{filtration}. Эти демоны
разрабатывались параллельно с данной работой и в её рамках не
рассматривались детально, однако их функциональность важна для
полноценного функционирования всей системы.

Следующим этапом обработки является \textbf{фильтрационный слой},
который принимает на вход собранные новости и выполняет интеллектуальную
фильтрацию, включая анализ содержания и пользовательские предпочтения.
Результаты работы фильтрационного слоя сохраняются в базе данных и
формируют коллекцию новостей, готовых к отправке пользователям.

Последняя стадия обработки --- \textbf{доставка новостей конечным
пользователям}, которая реализуется с помощью \textbf{Telegram-бота}.
Бот периодически запрашивает через Web Api подготовленные сообщения из
коллекции \texttt{NewsToSend} и отправляет их в соответствующие каналы
или личные чаты в Telegram. Отправка производится с учётом
индивидуальных настроек пользователей, включая предпочтения по
тематикам, ключевым словам и режиму уведомлений.

\hypertarget{ux430ux440ux445ux438ux442ux435ux43aux442ux443ux440ux430-ux444ux438ux43bux44cux442ux440ux430ux446ux438ux43eux43dux43dux43eux433ux43e-ux441ux43bux43eux44f}{%
\subsection{3.3 Архитектура фильтрационного
слоя}\label{ux430ux440ux445ux438ux442ux435ux43aux442ux443ux440ux430-ux444ux438ux43bux44cux442ux440ux430ux446ux438ux43eux43dux43dux43eux433ux43e-ux441ux43bux43eux44f}}

Фильтрационный слой системы мониторинга новостей реализован как связка
специализированных микросервисов, объединённых единой очередью сообщений
и REST-интерфейсами для вызова вспомогательных компонентов. Основной
задачей фильтрационного слоя является интеллектуальная обработка
новостных сообщений: от базовой предобработки до персонализированной
фильтрации под конкретных пользователей.

Центральным элементом слоя является микросервис \textbf{Filtrator},
реализующий цепочку обработки новостей. Архитектурно он организован
вокруг следующих компонентов и взаимодействий:

\begin{itemize}
\tightlist
\item
  \textbf{Очередь входящих сообщений (filtration)} --- источник входных
  данных, поступающих от демонов парсинга. Filtrator подписывается на
  неё и обрабатывает каждое сообщение независимо, используя пул
  воркеров.\\
\item
  \textbf{Предобработка сообщений} включает:

  \begin{itemize}
  \tightlist
  \item
    проверку времени и корректности формата данных;\\
  \item
    дедупликацию по ID;\\
  \item
    векторизацию текста через вызов сервиса \textbf{TextProcessor};\\
  \item
    анализ тональности, также с использованием \textbf{TextProcessor};\\
  \item
    определение смысловых дубликатов с помощью векторного сравнения.\\
  \end{itemize}
\item
  \textbf{Очередь batched-классификации (filterMlQueue)} ---
  промежуточный буфер, накапливающий сообщения после предобработки.
  Батчи формируются по таймеру или при достижении заданного количества
  сообщений (по умолчанию 10).\\
\item
  \textbf{Микросервис Classificator} получает батчи новостей и
  возвращает значение вероятности широкой огласки для каждой. Вызов
  осуществляется синхронно через REST API с политикой повторных попыток
  в случае сетевых или внутренних ошибок.\\
\item
  \textbf{Финальная обработка} включает:

  \begin{itemize}
  \tightlist
  \item
    сохранение новостей в MongoDB и индексацию в ElasticSearch;\\
  \item
    персонализированную фильтрацию под конкретных пользователей (учёт
    ключевых слов, дедупликации, пользовательских настроек);\\
  \item
    формирование объектов в коллекции \texttt{NewsToSend}, с
    метаинформацией для Telegram-бота (включая флаг replyMessageId для
    формирования цепочек).
  \end{itemize}
\end{itemize}

Взаимодействие между компонентами осуществляется асинхронно (через
RabbitMQ) либо по REST (между Filtrator и вспомогательными сервисами).

Архитектура слоя представлена на рисунке (см. TODO).

\hypertarget{ux440ux435ux430ux43bux438ux437ux430ux446ux438ux44f-ux441ux435ux440ux432ux438ux441ux430-filtrator}{%
\subsection{3.4. Реализация сервиса
Filtrator}\label{ux440ux435ux430ux43bux438ux437ux430ux446ux438ux44f-ux441ux435ux440ux432ux438ux441ux430-filtrator}}

Микросервис \textbf{Filtrator} реализует основную логику фильтрационного
слоя и отвечает за обработку новостных сообщений, поступающих из очереди
\texttt{filtration}. Внутри сервиса реализовано два основных этапа
обработки: \textbf{предобработка (pre-process)} и \textbf{постобработка
с участием модели (ml-process)}.

\hypertarget{ux43fux440ux435ux434ux43eux431ux440ux430ux431ux43eux442ux43aux430}{%
\subsubsection{\texorpdfstring{\textbf{Предобработка}}{Предобработка}}\label{ux43fux440ux435ux434ux43eux431ux440ux430ux431ux43eux442ux43aux430}}

На первом этапе сервис подписывается на очередь \texttt{filtration} и
обрабатывает каждое входящее сообщение в отдельном воркере. Обработка
включает следующие действия:

\begin{enumerate}
\def\labelenumi{\arabic{enumi}.}
\tightlist
\item
  \textbf{Проверка времени публикации}: если формат времени некорректен,
  сообщение отбрасывается.\\
\item
  \textbf{Проверка на дублирование по идентификатору}: если новость с
  тем же ID уже существует в MongoDB, она также отбрасывается.\\
\item
  \textbf{Векторизация текста}: новость передаётся в микросервис
  \texttt{text\_processor}, где рассчитывается векторное представление
  текста. Этот вектор добавляется в объект новости.\\
\item
  \textbf{Определение эмоции}: осуществляется попытка получить
  эмоциональную окраску текста через тот же \texttt{text\_processor}. В
  случае ошибки значение по умолчанию --- \texttt{unknown}.\\
\item
  \textbf{Определение дубликатов по смыслу}: новость сравнивается с
  ранее обработанными сообщениями с помощью компонента
  \texttt{NewsFilterService}. Дедупликация производится с помощью
  векторного сравнения (knn-поиск) на основе предварительно
  рассчитанного вектора (см. п. 3). Если новость признана дубликатом, в
  неё добавляется ссылка на оригинал и соответствующая метка.\\
  Также планируется дальнейшее улучшение дедупликации с помощью
  машинного обучения, т.е. если новости векторно похожи, дополнительно
  сравниваем их с помощью предобученной модели.
\end{enumerate}

После завершения предобработки сообщение отправляется во вторую очередь
\texttt{filterMlQueue} для последующего этапа.

\hypertarget{ux431ux430ux442ux447ux438ux43dux433-ux438-ux43aux43bux430ux441ux441ux438ux444ux438ux43aux430ux446ux438ux44f}{%
\subsubsection{\texorpdfstring{\textbf{Батчинг и
классификация}}{Батчинг и классификация}}\label{ux431ux430ux442ux447ux438ux43dux433-ux438-ux43aux43bux430ux441ux441ux438ux444ux438ux43aux430ux446ux438ux44f}}

Сервис параллельно подписан на очередь \texttt{filterMlQueue}. Все
новости, поступившие на этом этапе, накапливаются в \textbf{батч
конфигурируемого размера} (по умолчанию 10 штук) или обрабатываются по
\textbf{таймауту}, если новые сообщения не поступают. Когда батч готов,
он передаётся в микросервис \textbf{Classificator}, где для каждой
новости рассчитывается \textbf{вероятность огласки} (показатель
\texttt{BangerProbability}). Классификация осуществляется через REST API
с политикой повторных попыток в случае ошибок.

\hypertarget{ux441ux43eux445ux440ux430ux43dux435ux43dux438ux435-ux438-ux444ux438ux43bux44cux442ux440ux430ux446ux438ux44f-ux43fux43e-ux43fux43eux43bux44cux437ux43eux432ux430ux442ux435ux43bux44fux43c}{%
\subsubsection{\texorpdfstring{\textbf{Сохранение и фильтрация по
пользователям}}{Сохранение и фильтрация по пользователям}}\label{ux441ux43eux445ux440ux430ux43dux435ux43dux438ux435-ux438-ux444ux438ux43bux44cux442ux440ux430ux446ux438ux44f-ux43fux43e-ux43fux43eux43bux44cux437ux43eux432ux430ux442ux435ux43bux44fux43c}}

После получения результата классификации каждая новость:

\begin{itemize}
\tightlist
\item
  сохраняется в коллекции \texttt{News} MongoDB;\\
\item
  индексируется в ElasticSearch;\\
\item
  проверяется по настройкам пользователей, подписанных на
  соответствующий источник.
\end{itemize}

Фильтрация выполняется с учётом:

\begin{itemize}
\tightlist
\item
  включена ли опция дедупликации;\\
\item
  наличие совпадений по ключевым словам;\\
\item
  другие параметры профиля пользователя.
\end{itemize}

Если новость удовлетворяет критериям, она сохраняется в коллекции
\texttt{NewsToSend}. Для дубликатов предусмотрен механизм ``ответа на
оригинал'': если оригинальная новость уже была отправлена пользователю,
новая помечается как обновление с \texttt{replyMessageId}, что позволяет
боту сформировать цепочку сообщений.

Таким образом, сервис \textbf{Filtrator} выполняет полную обработку
новостных сообщений от входной очереди до формирования пользовательских
задач на отправку, интегрируя в себе дедупликацию, машинное обучение и
персонализированную фильтрацию.

\hypertarget{ux43cux438ux43aux440ux43eux441ux435ux440ux432ux438ux441-textprocessor}{%
\subsection{3.5. Микросервис
TextProcessor}\label{ux43cux438ux43aux440ux43eux441ux435ux440ux432ux438ux441-textprocessor}}

Микросервис \textbf{TextProcessor} отвечает за векторизацию текста
новостных сообщений и определение их эмоциональной окраски. Он
используется на этапе предобработки в рамках фильтрационного слоя и
вызывается aсинхронно из микросервиса \textbf{Filtrator}.

\hypertarget{ux430ux440ux445ux438ux442ux435ux43aux442ux443ux440ux430-ux438-ux442ux435ux445ux43dux43eux43bux43eux433ux438ux438}{%
\subsubsection{\texorpdfstring{\textbf{Архитектура и
технологии}}{Архитектура и технологии}}\label{ux430ux440ux445ux438ux442ux435ux43aux442ux443ux440ux430-ux438-ux442ux435ux445ux43dux43eux43bux43eux433ux438ux438}}

Сервис реализован с использованием языка \textbf{Python} и фреймворка
\textbf{FastAPI}. Для векторизации применяется предобученная модель
\texttt{paraphrase-multilingual-MiniLM-L12-v2} из библиотеки
\textbf{sentence-transformers}, обеспечивающая получение универсального
векторного представления текста на разных языках, включая русский.

Для анализа тональности используется модель
\texttt{blanchefort/rubert-base-cased-sentiment}, предназначенная для
работы с русскоязычными текстами. Она классифицирует эмоциональную
окраску как \textbf{положительную}, \textbf{отрицательную} или
\textbf{нейтральную}, что позволяет проводить базовую эмоциональную
фильтрацию новостей.

Сервис предоставляет два HTTP-эндпоинта:

\begin{itemize}
\tightlist
\item
  \texttt{POST\ /vectorize} --- возвращает векторное представление
  текста;\\
\item
  \texttt{POST\ /emotion} --- возвращает эмоциональную окраску текста.
\end{itemize}

\hypertarget{ux432ux437ux430ux438ux43cux43eux434ux435ux439ux441ux442ux432ux438ux435-ux441-ux434ux440ux443ux433ux438ux43cux438-ux43aux43eux43cux43fux43eux43dux435ux43dux442ux430ux43cux438}{%
\subsubsection{\texorpdfstring{\textbf{Взаимодействие с другими
компонентами}}{Взаимодействие с другими компонентами}}\label{ux432ux437ux430ux438ux43cux43eux434ux435ux439ux441ux442ux432ux438ux435-ux441-ux434ux440ux443ux433ux438ux43cux438-ux43aux43eux43cux43fux43eux43dux435ux43dux442ux430ux43cux438}}

Сервис вызывается асинхронно из микросервиса \textbf{Filtrator} при
получении новой новости:

\begin{itemize}
\tightlist
\item
  сначала происходит векторизация текста;\\
\item
  затем определяется его тональность;\\
\item
  полученные данные записываются в объект новости и используются на
  следующих этапах обработки (дедупликация, фильтрация, классификация).
\end{itemize}

В ходе нагрузочного тестирования (см. п. TODO) было выявлено, что
одиночная обработка запросов на векторизацию и определение тональности
является узким местом. В дальнейшем планируется переход на батчевую
обработку.

Таким образом, \textbf{TextProcessor} выполняет ключевую роль в
подготовке новостей к дальнейшему анализу, обеспечивая семантическое
представление и базовую эмоциональную характеристику сообщений.

\hypertarget{ux43cux438ux43aux440ux43eux441ux435ux440ux432ux438ux441-classificator}{%
\subsection{3.6. Микросервис
Classificator}\label{ux43cux438ux43aux440ux43eux441ux435ux440ux432ux438ux441-classificator}}

Микросервис \textbf{Classificator} отвечает за определение вероятности
широкого распространения новостного сообщения --- метрики, обозначенной
как \textbf{вероятность огласки} (\emph{BangerProbability}). Он
используется после этапа предобработки и векторизации новостей, когда
они уже прошли первичный фильтр и накапливаются в батчи в микросервисе
\textbf{Filtrator}.

\hypertarget{ux430ux440ux445ux438ux442ux435ux43aux442ux443ux440ux430-ux438-ux43dux430ux437ux43dux430ux447ux435ux43dux438ux435}{%
\subsubsection{\texorpdfstring{\textbf{Архитектура и
назначение}}{Архитектура и назначение}}\label{ux430ux440ux445ux438ux442ux435ux43aux442ux443ux440ux430-ux438-ux43dux430ux437ux43dux430ux447ux435ux43dux438ux435}}

Сервис реализован как API-приложение на основе \textbf{FastAPI}.
Основная задача --- принимать батчи новостей и возвращать список
вероятностей, отражающих степень ``вирусности'' или потенциальной
значимости каждой публикации. Обработка осуществляется асинхронно по
HTTP-протоколу.

\hypertarget{ux43cux43eux434ux435ux43bux438-ux438-ux43eux431ux440ux430ux431ux43eux442ux43aux430}{%
\subsubsection{\texorpdfstring{\textbf{Модели и
обработка}}{Модели и обработка}}\label{ux43cux43eux434ux435ux43bux438-ux438-ux43eux431ux440ux430ux431ux43eux442ux43aux430}}

Разработка и обучение моделей машинного обучения в рамках данного
микросервиса \textbf{не входила в зону ответственности автора}.\\
В данной работе были реализованы:

\begin{enumerate}
\def\labelenumi{\arabic{enumi}.}
\tightlist
\item
  Сбор данных для обучения модели на основе ранее обработанных
  новостей\\
\item
  Инфраструктура API, обеспечивающая:

  \begin{itemize}
  \tightlist
  \item
    приём входных данных (тексты новостей или их векторные
    представления);\\
  \item
    вызов предобученной модели и получение прогнозов;\\
  \item
    возврат предсказаний в виде вероятностей от 0 до 1 (интерпретируемых
    как процент вероятности огласки).
  \end{itemize}
\end{enumerate}

Определение вероятности огласки выполняется на основе следующих данных:

\begin{itemize}
\tightlist
\item
  Текст публикации\\
\item
  Уровень влияния источника, рассчитывается на основе количества
  подписчиков и территориального уровня (окружной, региональный,
  федеральный)
\end{itemize}

Основной рабочий эндпоинт:

\begin{itemize}
\tightlist
\item
  \texttt{POST\ /predict\_batch} --- принимает список новостей и
  возвращает массив значений \texttt{probs}, соответствующих
  вероятностям распространения для каждой новости.
\end{itemize}

Взаимодействие с Filtrator-ом происходит асинхронно: батчи формируются
на стороне фильтрационного слоя и отправляются в Classificator по мере
достижения заданного размера или по таймауту.

\hypertarget{ux432ux441ux442ux440ux43eux435ux43dux43dux44bux435-ux43cux435ux445ux430ux43dux438ux437ux43cux44b}{%
\subsubsection{\texorpdfstring{\textbf{Встроенные
механизмы}}{Встроенные механизмы}}\label{ux432ux441ux442ux440ux43eux435ux43dux43dux44bux435-ux43cux435ux445ux430ux43dux438ux437ux43cux44b}}

Сервис включает:

\begin{itemize}
\tightlist
\item
  базовую валидацию входных данных;\\
\item
  регистрацию ошибок с понятными HTTP-ответами;\\
\item
  фоновые задачи (например, отложенное дообучение через
  \texttt{BatchAccumulator}, если потребуется доработка модели в
  будущем).
\end{itemize}

Таким образом, Classificator обеспечивает масштабируемую точку входа для
применения моделей оценки новостей и интегрируется с фильтрационным
слоем в рамках общей архитектуры микросервисной системы.

\hypertarget{ux440ux430ux437ux432ux451ux440ux442ux44bux432ux430ux43dux438ux435-ux438-ux43eux431ux435ux441ux43fux435ux447ux435ux43dux438ux435-ux43eux442ux43aux430ux437ux43eux443ux441ux442ux43eux439ux447ux438ux432ux43eux441ux442ux438}{%
\section{4. Развёртывание и обеспечение
отказоустойчивости}\label{ux440ux430ux437ux432ux451ux440ux442ux44bux432ux430ux43dux438ux435-ux438-ux43eux431ux435ux441ux43fux435ux447ux435ux43dux438ux435-ux43eux442ux43aux430ux437ux43eux443ux441ux442ux43eux439ux447ux438ux432ux43eux441ux442ux438}}

\hypertarget{ux43eux431ux449ux430ux44f-ux441ux445ux435ux43cux430-ux440ux430ux437ux432ux451ux440ux442ux44bux432ux430ux43dux438ux44f}{%
\subsection{4.1. Общая схема
развёртывания}\label{ux43eux431ux449ux430ux44f-ux441ux445ux435ux43cux430-ux440ux430ux437ux432ux451ux440ux442ux44bux432ux430ux43dux438ux44f}}

Система мониторинга новостных ресурсов развёрнута в виде микросервисной
архитектуры с применением Docker Swarm. Все компоненты упакованы в
отдельные контейнеры, что обеспечивает изоляцию, гибкость конфигурации и
масштабируемость. Основные микросервисы включают:

\begin{itemize}
\tightlist
\item
  \texttt{filtrator} --- основной фильтрационный слой;\\
\item
  \texttt{text\_processor} --- сервис векторизации текста и определения
  эмоции;\\
\item
  \texttt{classificator} --- сервис определения вероятности широкого
  распространения новости;\\
\item
  \texttt{vk\_daemon}, \texttt{rss\_daemon}, \texttt{telegram\_parser},
  \texttt{telegram\_subscriber}, \texttt{master\_daemon} --- демоны,
  получающие данные из внешних источников;\\
\item
  \texttt{webapi} --- API-интерфейс;\\
\item
  \texttt{tg\_bot} --- Telegram-бот, отвечающий за отправку новостей
  пользователю.
\end{itemize}

Также в состав системы входят:

\begin{itemize}
\tightlist
\item
  MongoDB;\\
\item
  RabbitMQ;\\
\item
  ElasticSearch + Kibana;\\
\item
  Filebeat для сбора логов.
\end{itemize}

Развёртывание выполнено на домашнем сервере с Ubuntu, подключённом к
облачному VPS по WireGuard, что обеспечивает внешний доступ к внутренней
инфраструктуре без необходимости статического IP-адреса.

\hypertarget{ux43dux430ux441ux442ux440ux43eux439ux43aux430-docker-swarm}{%
\subsection{4.2. Настройка Docker
Swarm}\label{ux43dux430ux441ux442ux440ux43eux439ux43aux430-docker-swarm}}

Контейнеры управляются с помощью Docker Swarm. Это позволило:

\begin{itemize}
\tightlist
\item
  обеспечить оркестрацию микросервисов;\\
\item
  централизованно управлять конфигурацией и переменными окружения;\\
\item
  использовать общие \texttt{docker-compose} файлы с возможностью
  переопределения хостов, портов и путей.
\end{itemize}

Для повышения устойчивости контейнеров используется политика
\texttt{restart:\ always}. Однако в перспективе планируется переход на
более надёжный механизм --- \textbf{healthcheck-мониторинг}, который
позволит выявлять не только падения контейнеров, но и внутренние ошибки
приложения, влияющие на его корректную работу.

\hypertarget{ux43dux430ux441ux442ux440ux43eux439ux43aux430-mongodb}{%
\subsection{4.3. Настройка
MongoDB}\label{ux43dux430ux441ux442ux440ux43eux439ux43aux430-mongodb}}

MongoDB изначально разворачивалась в режиме репликации (ReplicaSet), с
полным набором скриптов и \texttt{keyfile}-аутентификацией. Однако в
силу ограничений по ресурсам репликация была временно отключена.

При этом:

\begin{itemize}
\tightlist
\item
  подготовлена инфраструктура для быстрого возврата к отказоустойчивому
  режиму;\\
\item
  все коллекции (пользователи, новости, подписки и т.д.) создаются
  автоматически;\\
\item
  добавлены необходимые индексы (например, по \texttt{timestamp},
  \texttt{vector}, \texttt{user\_id}, \texttt{status}), повышающие
  производительность запросов и устойчивость системы к нагрузкам.
\end{itemize}

\hypertarget{ux43dux430ux441ux442ux440ux43eux439ux43aux430-rabbitmq}{%
\subsection{4.4. Настройка
RabbitMQ}\label{ux43dux430ux441ux442ux440ux43eux439ux43aux430-rabbitmq}}

RabbitMQ развёрнут в виде одиночного экземпляра с сохранением состояния
через volume. Все очереди создаются автоматически с помощью специального
bash-скрипта, который:

\begin{itemize}
\tightlist
\item
  создаёт пользователя и виртуальный хост;\\
\item
  настраивает права доступа;\\
\item
  инициализирует очереди (\texttt{news\_queue\_vk},
  \texttt{news\_queue\_rss}, \texttt{news\_queue\_tg},
  \texttt{filter\_queue}, \texttt{send\_queue}, \texttt{ack\_queue} и
  др.).
\end{itemize}

В будущем возможен переход к кластерной конфигурации RabbitMQ для
обеспечения отказоустойчивости и балансировки нагрузки.

\hypertarget{ux43dux430ux441ux442ux440ux43eux439ux43aux430-elasticsearch-kibana-ux438-filebeat}{%
\subsection{4.5. Настройка ElasticSearch, Kibana и
Filebeat}\label{ux43dux430ux441ux442ux440ux43eux439ux43aux430-elasticsearch-kibana-ux438-filebeat}}

ElasticSearch используется для хранения и поиска новостей и логов.
Настроена работа с TLS-сертификатами, сгенерированными вручную. Kibana
предоставляет визуальный интерфейс для работы с данными.

Filebeat конфигурирован на сбор логов из всех контейнеров Docker,
включая:

\begin{itemize}
\tightlist
\item
  структурирование логов (декодирование JSON, парсинг полей);\\
\item
  фильтрацию лишних сообщений (например, от MongoDB);\\
\item
  автоматическую разбивку логов по индексам:
  \texttt{logs-\{service\_name\}-prod-\{date\}}.
\end{itemize}

Всё это обеспечивает удобную отладку и анализ системы в реальном
времени.

Потенциал развития:

\begin{itemize}
\tightlist
\item
  подключение Alerting-механизмов в Kibana;\\
\item
  внедрение Elastic APM для трассировки вызовов между сервисами.
\end{itemize}

\hypertarget{ux43eux431ux435ux441ux43fux435ux447ux435ux43dux438ux435-ux43eux442ux43aux430ux437ux43eux443ux441ux442ux43eux439ux447ux438ux432ux43eux441ux442ux438}{%
\subsection{4.6. Обеспечение
отказоустойчивости}\label{ux43eux431ux435ux441ux43fux435ux447ux435ux43dux438ux435-ux43eux442ux43aux430ux437ux43eux443ux441ux442ux43eux439ux447ux438ux432ux43eux441ux442ux438}}

Были предприняты следующие шаги:

\begin{itemize}
\tightlist
\item
  Внедрены \textbf{ретрай-политики} (через Polly и собственные
  механизмы) при обращении к внешним сервисам: MongoDB, RabbitMQ,
  TextProcessor, Classificator.\\
\item
  Используется \textbf{batch-обработка} сообщений, позволяющая
  сглаживать нагрузку и минимизировать потери при сбоях.\\
\item
  Весь обмен сообщениями между сервисами организован через
  \textbf{устойчивые очереди} RabbitMQ (\texttt{durable:\ true}).\\
\item
  Добавлено \textbf{обработка ошибок и логирование} в ключевых точках.\\
\item
  Используется \textbf{асинхронная модель} выполнения в Python-сервисах
  (RSS, Telegram, TextProcessor), что обеспечивает высокую отзывчивость
  при одновременной работе с десятками источников.
\end{itemize}

В будущем планируется интеграция \textbf{healthcheck-механизмов},
позволяющих более точно управлять перезапуском контейнеров и
обнаружением «тихих сбоев».

\hypertarget{ux43bux43eux433ux438ux440ux43eux432ux430ux43dux438ux435-ux438-ux43cux43eux43dux438ux442ux43eux440ux438ux43dux433}{%
\subsection{4.7. Логирование и
мониторинг}\label{ux43bux43eux433ux438ux440ux43eux432ux430ux43dux438ux435-ux438-ux43cux43eux43dux438ux442ux43eux440ux438ux43dux433}}

Система логирования реализована на базе Filebeat + Elastic + Kibana. Все
логи собираются в формате JSON, что обеспечивает возможность
автоматической агрегации и анализа.

На данный момент отсутствует метрик-ориентированный мониторинг
(Prometheus, Grafana), но архитектура системы предусматривает лёгкое
подключение данных инструментов в будущем.

\hypertarget{ux433ux438ux431ux43aux430ux44f-ux43aux43eux43dux444ux438ux433ux443ux440ux430ux446ux438ux44f-ux441ux435ux440ux432ux438ux441ux43eux432}{%
\subsection{4.8. Гибкая конфигурация
сервисов}\label{ux433ux438ux431ux43aux430ux44f-ux43aux43eux43dux444ux438ux433ux443ux440ux430ux446ux438ux44f-ux441ux435ux440ux432ux438ux441ux43eux432}}

Особенностью реализации является удобная система конфигурации:

\begin{itemize}
\tightlist
\item
  Все переменные окружения вынесены в \texttt{.env} и
  \texttt{hosts.env}, что позволяет быстро переключаться между
  окружениями (dev/prod).\\
\item
  Возможен \textbf{гибридный режим} разработки --- часть сервисов
  запускается в Docker, часть --- напрямую из IDE. Это существенно
  ускоряет отладку.\\
\item
  Все пути к основным сервисам (Mongo, Elastic, RabbitMQ, WebAPI)
  задаются параметрами, что упрощает развертывание в различных
  инфраструктурах.
\end{itemize}

\hypertarget{ux431ux435ux437ux43eux43fux430ux441ux43dux43eux441ux442ux44c}{%
\subsection{4.9.
Безопасность}\label{ux431ux435ux437ux43eux43fux430ux441ux43dux43eux441ux442ux44c}}

Для обеспечения безопасности предприняты следующие шаги:

\begin{itemize}
\tightlist
\item
  Все соединения с ElasticSearch и Kibana защищены с помощью
  SSL-сертификатов.\\
\item
  Используется WireGuard для подключения внутреннего домашнего сервера к
  облачному VPS, обеспечивая защищённый канал доступа к сервисам.\\
\item
  Для RabbitMQ и MongoDB настроены отдельные пользователи и роли с
  минимальными правами.
\end{itemize}

В будущем планируется реализация:

\begin{itemize}
\tightlist
\item
  ограничений по IP;\\
\item
  централизованного управления правами доступа;
\end{itemize}

\hypertarget{ux442ux435ux441ux442ux438ux440ux43eux432ux430ux43dux438ux435-ux443ux441ux442ux43eux439ux447ux438ux432ux43eux441ux442ux438}{%
\subsection{4.10. Тестирование
устойчивости}\label{ux442ux435ux441ux442ux438ux440ux43eux432ux430ux43dux438ux435-ux443ux441ux442ux43eux439ux447ux438ux432ux43eux441ux442ux438}}

В рамках разработки были проведены следующие виды тестирования:

\begin{itemize}
\tightlist
\item
  перезапуск отдельных контейнеров и всей системы в целом;\\
\item
  отключение ключевых компонентов (\texttt{TextProcessor},
  \texttt{Classificator}) с последующим восстановлением;\\
\item
  стресс-тестирование системы с симуляцией массового поступления
  новостей. TODO расписать
\end{itemize}

Результаты подтвердили устойчивость системы и её способность к
восстановлению без потерь.

\hypertarget{ux437ux430ux43aux43bux44eux447ux435ux43dux438ux435}{%
\section{Заключение}\label{ux437ux430ux43aux43bux44eux447ux435ux43dux438ux435}}

В рамках выпускной квалификационной работы была разработана и частично
внедрена система мониторинга новостных ресурсов, ориентированная на
высокую надёжность, масштабируемость и качество фильтрации контента.
Основное внимание уделялось созданию \textbf{фильтрационного слоя},
обеспечивающего интеллектуальную предобработку новостных сообщений, и
построению \textbf{инфраструктуры}, устойчивой к сбоям.

В ходе реализации достигнуты следующие результаты:

\begin{itemize}
\tightlist
\item
  Разработан фильтрационный микросервис \texttt{Filtrator}, включающий:

  \begin{itemize}
  \tightlist
  \item
    предобработку новостей и асинхронную векторизацию текста;\\
  \item
    анализ эмоций и определение вероятности ``вирусности'' публикации;\\
  \item
    дедупликацию на основе векторных представлений текста;\\
  \item
    фильтрацию по пользовательским настройкам и распределение
    новостей.\\
  \end{itemize}
\item
  Разработаны и внедрены два вспомогательных микросервиса:

  \begin{itemize}
  \tightlist
  \item
    \texttt{TextProcessor} --- для векторизации и анализа тональности;\\
  \item
    \texttt{Classificator} --- для оценки вероятности широкого
    распространения новостей.\\
  \end{itemize}
\item
  Вся система развёрнута с использованием \textbf{Docker Swarm}, с
  удобной, параметризуемой конфигурацией, поддержкой локальной отладки и
  потенциальной миграцией в облако.\\
\item
  Реализована начальная поддержка отказоустойчивости:

  \begin{itemize}
  \tightlist
  \item
    \texttt{retry}-политики в критичных местах;\\
  \item
    батч-обработка данных;\\
  \item
    \texttt{restart:\ always} в Docker;\\
  \item
    устойчивые очереди и хранение промежуточных результатов в MongoDB.\\
  \end{itemize}
\item
  Настроено централизованное логирование через Filebeat и
  ElasticSearch\\
\item
  Обеспечена базовая безопасность с помощью TLS и WireGuard.
\end{itemize}

В ходе работы было выявлено несколько направлений для развития:

\begin{itemize}
\tightlist
\item
  внедрение полноценного мониторинга состояния микросервисов
  (Prometheus, Grafana, healthcheck);\\
\item
  автоматизация CI/CD и деплоймента в облачные инфраструктуры;\\
\item
  масштабирование брокера RabbitMQ и базы MongoDB с включением
  репликации;\\
\item
  улучшение качества классификации тональности и расширение
  фильтрационного слоя (включение Named Entity Recognition, тематической
  категоризации и т.д.).
\end{itemize}

Разработанная архитектура и фильтрационный слой подтвердили свою
эффективность в тестовых условиях и могут быть использованы в системах
анализа репутации, информационных панелях и корпоративных решениях для
обработки новостей и сообщений из открытых источников.

Таким образом, поставленные цели и задачи выпускной работы были
достигнуты, а созданная система обладает высоким потенциалом для
дальнейшего развития и промышленного применения.
