\newpage

{\centering
\normal
\textbf{ОСНОВНАЯ ЧАСТЬ}\par
}
\addcontentsline{toc}{chapter}{ОСНОВНАЯ ЧАСТЬ}

\customsection{Анализ методов фильтрации новостного контента и построения отказоустойчивых систем}

\customsubsection{Анализ существующих решений мониторинга новостного контента}\label{subsec:anal11}

\customsubsubsection{Дедупликация}\label{subsubsec:anal111}
Дедупликация — ключевой этап в обработке новостного контента, направленный на устранение повторяющихся или схожих сообщений.
Существуют различные методы дедупликации:
\begin{itemize}
    \item \textbf{Хеширование}: использование хеш-функций для идентификации идентичных текстов.
    \item \textbf{Сравнение по ключевым словам}: выделение и сравнение ключевых слов в текстах для определения схожести.
    \item \textbf{Семантический анализ}: применение методов обработки естественного языка для выявления смысловой близости текстов.
\end{itemize}

Современные исследования предлагают адаптивные методы дедупликации, учитывающие контекст и структуру данных, что повышает точность фильтрации.

\customsubsubsection{Фильтрация по ключевым словам}\label{subsubsec:anal112}
Фильтрация по ключевым словам является основным инструментом для отбора релевантного новостного контента.
Существуют различные подходы:
\begin{itemize}
    \item \textbf{Прямое сопоставление}: поиск точных совпадений ключевых слов в тексте.
    \item \textbf{Использование регулярных выражений}: позволяет учитывать различные формы слов и фраз.
    \item \textbf{Семантический анализ}: учет синонимов и контекста для более гибкой фильтрации.
\end{itemize}

Инструменты, такие как Elasticsearch, предоставляют мощные возможности для реализации фильтрации по ключевым словам с использованием различных методов анализа текста.

\customsubsubsection{Определение эмоционального окраса}\label{subsubsec:anal113}
Анализ эмоционального окраса (сентимент-анализ) позволяет классифицировать новостные сообщения по тональности: положительной, отрицательной или нейтральной.
Существуют различные методы:
\begin{itemize}
    \item \textbf{Словарные методы}: использование заранее составленных словарей с оценкой эмоциональной нагрузки слов.
    \item \textbf{Машинное обучение}: обучение моделей на размеченных данных для определения тональности.
    \item \textbf{Гибридные подходы}: сочетание словарных методов и машинного обучения для повышения точности.
\end{itemize}

Современные исследования подчеркивают эффективность гибридных методов, особенно при анализе коротких текстов, таких как заголовки новостей.

\customsubsubsection{Вычисление потенциальной вирусности}\label{subsubsec:anal114}
Оценка потенциальной вирусности новостного контента представляет собой задачу предсказания вероятности быстрого и широкого распространения материала в информационной среде. Для этого используются методы из областей машинного обучения, анализа графов и анализа текстов.

Наиболее распространённые подходы включают:
\begin{itemize}
    \item \textbf{Регрессионные модели (Logistic Regression, Random Forest)}: Используются для предсказания бинарного события — станет ли материал вирусным или нет, на основе метаданных (время публикации, количество слов, источник, первые реакции). Источник: Ma, J., Gao, W., \& Wong, K.-F. (2018). Rumor detection on Twitter with tree-structured recursive neural networks. ACL 2018.
    \item \textbf{Глубокие нейронные сети (CNN, RNN, Transformer)}: Модели, такие как BERT или LSTM, обучаются на текстах новостей с учетом их предыдущей популярности (например, число репостов или лайков). Это позволяет учитывать не только содержание, но и контекст публикации. Источник: Nguyen, D. T., Sugiyama, K., Nakov, P., \& Kan, M.-Y. (2020). FANG: Leveraging social context for fake news detection using graph representation. CIKM.
    \item \textbf{Графовые модели распространения (Graph Propagation Models)}: Контент рассматривается как узел в сети, а пользователи и связи между ними — как ребра. Применяются модели, такие как Independent Cascade или Linear Threshold для симуляции распространения. Источник: Kempe, D., Kleinberg, J., \& Tardos, É. (2003). Maximizing the spread of influence through a social network. KDD.
    \item \textbf{Векторизация контента с обучением на исторических данных (TF-IDF, Doc2Vec, BERT embeddings)}: Содержимое новости преобразуется в вектор признаков, и на его основе оценивается схожесть с уже известными вирусными публикациями. Источник: Bandari, R., Asur, S., \& Huberman, B. A. (2012). The Pulse of News in Social Media: Forecasting Popularity. ICWSM.
    \item \textbf{Гибридные модели (мультимодальные)}: Используются одновременно текст, метаданные и поведенческие данные (время просмотра, вовлечённость). Такие модели комбинируют разные типы входных признаков и обычно реализуются на основе ансамблей или нейросетей с несколькими входами. Источник: Tatar, A., Antoniadis, P., De Amorim, M. D., \& Fdida, S. (2014). A Survey on Predicting the Popularity of Web Content. Computer Communications, 36(11-12), 1132-1144.
\end{itemize}

Таким образом, вычисление вирусности представляет собой комплексную задачу, в которой используются как традиционные алгоритмы машинного обучения, так и современные методы анализа графов и нейросетевые архитектуры. Выбор подхода зависит от доступных данных и требуемой точности модели.

\customsubsection{Анализ существующих подходов к построению отказоустойчивых систем}\label{subsec:anal12}
Отказоустойчивость — способность системы продолжать функционировать при возникновении сбоев.
Современные подходы к построению отказоустойчивых систем включают:
\begin{itemize}
    \item \textbf{Микросервисная архитектура}: разделение системы на независимые сервисы, что позволяет локализовать сбои и облегчает масштабирование.
    \item \textbf{Использование оркестраторов}: инструменты, такие как Docker Swarm, управляют развертыванием, масштабированием и восстановлением сервисов.
    \item \textbf{Паттерны отказоустойчивости}: применение шаблонов проектирования, таких как Circuit Breaker и Retry, для обработки сбоев.
\end{itemize}

\customsubsection{Анализ существующих решений сбора телеметрии системы и хранения логов}\label{subsec:anal13}
Эффективный сбор телеметрии и логов необходим для мониторинга и диагностики систем.
Современные решения включают:
\begin{itemize}
    \item \textbf{Системы сбора метрик}: инструменты, такие как Prometheus, собирают и хранят метрики производительности.
    \item \textbf{Системы агрегации логов}: инструменты, такие как Filebeat, собирают, обрабатывают и передают логи в хранилища.
    \item \textbf{Платформы визуализации}: инструменты, такие как Kibana, предоставляют визуальное представление метрик и логов для анализа.
\end{itemize}

Интеграция этих инструментов обеспечивает полную наблюдаемость системы, позволяя оперативно выявлять и устранять проблемы.

\customsubsection{Выводы по главе 1}
В ходе анализа существующих решений в области мониторинга новостного контента и построения отказоустойчивых систем были сделаны следующие выводы:
\begin{enumerate}
    \item Методы фильтрации контента демонстрируют значительное разнообразие как по глубине анализа, так и по вычислительным затратам. Простейшие подходы, такие как хеширование и фильтрация по ключевым словам, обеспечивают высокую производительность, но ограничены в выявлении смысловых дубликатов или релевантных публикаций в изменённой формулировке. Более продвинутые методы — семантический анализ и гибридные модели — требуют больших вычислительных ресурсов, но дают более точные результаты и способны учитывать контекст.
    \item Определение эмоционального окраса и оценка вирусности основаны на широком спектре методов: от словарных и регрессионных до графовых и нейросетевых. Эффективное использование этих методов позволяет не только фильтровать контент по тону, но и прогнозировать его дальнейшее распространение в сети, что критично для новостного мониторинга в высоконагруженных информационных средах.
    \item Современные подходы к построению отказоустойчивых систем, такие как микросервисная архитектура в сочетании с Docker Swarm, обеспечивают гибкость масштабирования и изоляцию сбоев. Использование паттернов обработки ошибок, включая политику повторных попыток (retry), позволяет значительно повысить надёжность работы сервисов.
    \item Системы наблюдаемости и логирования, основанные на связке Filebeat, Elasticsearch и Kibana, являются промышленным стандартом для обеспечения прозрачности работы компонентов системы. Они позволяют своевременно выявлять отклонения в поведении сервисов и проводить диагностику на основе собранных логов и метрик.
\end{enumerate}

Таким образом, анализ показал, что эффективная система мониторинга новостного контента должна сочетать в себе современные методы фильтрации, механизмы предсказания распространения информации, архитектурные решения для отказоустойчивости и развитую систему телеметрии. Эти аспекты легли в основу проектирования и реализации предлагаемого решения, рассмотренного в следующих главах работы.
