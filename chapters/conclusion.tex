\newpage
{\centering
\normal
\textbf{ЗАКЛЮЧЕНИЕ}\par
}
\addcontentsline{toc}{chapter}{ЗАКЛЮЧЕНИЕ}

В рамках выпускной квалификационной работы была разработана и частично внедрена система мониторинга новостных ресурсов, ориентированная на высокую надёжность, масштабируемость и качество фильтрации контента.
Основное внимание уделялось созданию \textbf{фильтрационного слоя}, обеспечивающего интеллектуальную предобработку новостных сообщений, и построению \textbf{инфраструктуры}, устойчивой к сбоям.

В ходе реализации достигнуты следующие результаты:
\begin{itemize}
  \item Разработан фильтрационный микросервис \textbf{Filtrator}, включающий:
  \begin{itemize}
    \item предобработку новостей и асинхронную векторизацию текста;
    \item анализ эмоций и определение вероятности "вирусности" публикации;
    \item дедупликацию на основе векторных представлений текста;
    \item фильтрацию по пользовательским настройкам и распределение новостей.
  \end{itemize}
  \item Разработаны и внедрены два вспомогательных микросервиса:
  \begin{itemize}
    \item \textbf{TextProcessor} — для векторизации и анализа тональности;
    \item \textbf{Classificator} — для оценки вероятности широкого распространения новостей.
  \end{itemize}
  \item Вся система развёрнута с использованием \textbf{Docker Swarm}, с удобной, параметризуемой конфигурацией, поддержкой локальной отладки и потенциальной миграцией в облако.
  \item Реализована начальная поддержка отказоустойчивости:
  \begin{itemize}
    \item \texttt{retry}-политики в критичных местах;
    \item \textbf{батч}-обработка данных;
    \item \texttt{restart: always} в Docker;
    \item устойчивые очереди и хранение промежуточных результатов в MongoDB.
  \end{itemize}
  \item Настроено централизованное логирование через Filebeat и ElasticSearch
  \item Обеспечена базовая безопасность с помощью TLS и WireGuard.
\end{itemize}

В ходе работы было выявлено несколько направлений для развития:
\begin{itemize}
  \item внедрение полноценного мониторинга состояния микросервисов (Prometheus, Grafana, healthcheck);
  \item автоматизация CI/CD и деплоймента в облачные инфраструктуры;
  \item масштабирование брокера RabbitMQ и базы MongoDB с включением репликации;
  \item улучшение качества классификации тональности и расширение фильтрационного слоя (включение Named Entity Recognition, тематической категоризации и т.д.).
\end{itemize}

Разработанная архитектура и фильтрационный слой подтвердили свою эффективность в тестовых условиях и могут быть использованы в системах анализа репутации, информационных панелях и корпоративных решениях для обработки новостей и сообщений из открытых источников.
Таким образом, поставленные цели и задачи выпускной работы были достигнуты, а созданная система обладает высоким потенциалом для дальнейшего развития и промышленного применения.

