\newpage

\customsection{Развёртывание и обеспечение отказоустойчивости}

\customsubsection{Общая схема развёртывания}
Система мониторинга новостных ресурсов развёрнута в виде микросервисной архитектуры с применением Docker Swarm.
Все компоненты упакованы в отдельные контейнеры, что обеспечивает изоляцию, гибкость конфигурации и масштабируемость.
Основные микросервисы включают:
\begin{itemize}
  \item \texttt{filtrator} — основной фильтрационный слой;
  \item \texttt{text\_processor} — сервис векторизации текста и определения эмоции;
  \item \texttt{classificator} — сервис определения вероятности широкого распространения новости;
  \item \texttt{vk\_daemon}, \texttt{rss\_daemon}, \texttt{telegram\_parser}, \texttt{telegram\_subscriber}, \texttt{master\_daemon} — демоны, получающие данные из внешних источников;
  \item \texttt{webapi} — API-интерфейс;
  \item \texttt{tg\_bot} — Telegram-бот, отвечающий за отправку новостей пользователю.
\end{itemize}

Также в состав системы входят:
\begin{itemize}
  \item MongoDB;
  \item RabbitMQ;
  \item ElasticSearch + Kibana;
  \item Filebeat для сбора логов.
\end{itemize}

Развёртывание выполнено на домашнем сервере с Ubuntu, подключённом к облачному VPS по WireGuard, что обеспечивает внешний доступ к внутренней инфраструктуре без необходимости статического IP-адреса.

\customsubsection{Настройка Docker Swarm}
Контейнеры управляются с помощью Docker Swarm.
Это позволило:
\begin{itemize}
  \item обеспечить оркестрацию микросервисов;
  \item централизованно управлять конфигурацией и переменными окружения;
  \item использовать общие \texttt{docker-compose} файлы с возможностью переопределения хостов, портов и путей.
\end{itemize}

Для повышения устойчивости контейнеров используется политика \texttt{restart: always}.
Однако в перспективе планируется переход на более надёжный механизм — \texttt{healthcheck}-мониторинг, который позволит выявлять не только падения контейнеров, но и внутренние ошибки приложения, влияющие на его корректную работу.

\customsubsection{Настройка MongoDB}
MongoDB изначально разворачивалась в режиме репликации (ReplicaSet), с полным набором скриптов и \texttt{keyfile}-аутентификацией.
Однако в силу ограничений по ресурсам репликация была временно отключена.

При этом:
\begin{itemize}
  \item подготовлена инфраструктура для быстрого возврата к отказоустойчивому режиму;
  \item все коллекции (пользователи, новости, подписки и т.д.) создаются автоматически;
  \item добавлены необходимые индексы (например, по \texttt{timestamp}, \texttt{vector}, \texttt{user\_id}, \texttt{status}), повышающие производительность запросов и устойчивость системы к нагрузкам.
\end{itemize}

\customsubsection{Настройка RabbitMQ}
RabbitMQ развёрнут в виде одиночного экземпляра с сохранением состояния через \texttt{volume}.
Все очереди создаются автоматически с помощью специального \texttt{bash}-скрипта, который:
\begin{itemize}
  \item создаёт пользователя и виртуальный хост;
  \item настраивает права доступа;
  \item инициализирует очереди (\texttt{news\_queue\_vk}, \texttt{news\_queue\_rss}, \texttt{filter\_queue}).
\end{itemize}

В будущем возможен переход к кластерной конфигурации RabbitMQ для обеспечения отказоустойчивости и балансировки нагрузки.

\customsubsection{Настройка ElasticSearch, Kibana и Filebeat}
ElasticSearch используется для хранения и поиска новостей и логов.
Настроена работа с TLS-сертификатами, сгенерированными вручную.
Kibana предоставляет визуальный интерфейс для работы с данными.

Filebeat конфигурирован на сбор логов из всех контейнеров Docker, включая:
\begin{itemize}
  \item структурирование логов (декодирование JSON, парсинг полей);
  \item фильтрацию лишних сообщений (например, от MongoDB);
  \item автоматическую разбивку логов по индексам: \texttt{logs-{service\_name}-prod-{date}}.
\end{itemize}

Всё это обеспечивает удобную отладку и анализ системы в реальном времени.

Потенциал развития:
\begin{itemize}
  \item подключение Alerting-механизмов в Kibana;
  \item внедрение Elastic APM для трассировки вызовов между сервисами.
\end{itemize}

\customsubsection{Обеспечение отказоустойчивости}
Были предприняты следующие шаги:
\begin{itemize}
  \item Внедрены \textbf{ретрай}-политики (через Polly и собственные механизмы) при обращении к внешним сервисам: MongoDB, RabbitMQ, TextProcessor, Classificator.
  \item Используется \texttt{batch}-обработка сообщений, позволяющая сглаживать нагрузку и минимизировать потери при сбоях.
  \item Весь обмен сообщениями между сервисами организован через \textbf{устойчивые очереди} RabbitMQ (\texttt{durable: true}).
  \item Добавлено \textbf{обработка ошибок и логирование} в ключевых точках.
  \item Используется \textbf{асинхронная модель} выполнения в Python-сервисах (RSS, Telegram, TextProcessor), что обеспечивает высокую отзывчивость при одновременной работе с десятками источников.
\end{itemize}

В будущем планируется интеграция \texttt{healthcheck}-механизмов, позволяющих более точно управлять перезапуском контейнеров и обнаружением «тихих сбоев».

\customsubsection{Логирование и мониторинг}
Система логирования реализована на базе Filebeat + Elastic + Kibana.
Все логи собираются в формате JSON, что обеспечивает возможность автоматической агрегации и анализа.

На данный момент отсутствует метрик-ориентированный мониторинг (Prometheus, Grafana), но архитектура системы предусматривает лёгкое подключение данных инструментов в будущем.

\customsubsection{Гибкая конфигурация сервисов}
Особенностью реализации является удобная система конфигурации:
\begin{itemize}
  \item Все переменные окружения вынесены в \texttt{.env} и \texttt{hosts.env}, что позволяет быстро переключаться между окружениями (dev/prod).
  \item Возможен \textbf{гибридный режим} разработки — часть сервисов запускается в Docker, часть — напрямую из IDE. Это существенно ускоряет отладку.
  \item Все пути к основным сервисам (Mongo, Elastic, RabbitMQ, WebAPI) задаются параметрами, что упрощает развертывание в различных инфраструктурах.
\end{itemize}

\customsubsection{Безопасность}
Для обеспечения безопасности предприняты следующие шаги:
\begin{itemize}
  \item Все соединения с ElasticSearch и Kibana защищены с помощью SSL-сертификатов.
  \item Используется WireGuard для подключения внутреннего домашнего сервера к облачному VPS, обеспечивая защищённый канал доступа к сервисам.
  \item Для RabbitMQ и MongoDB настроены отдельные пользователи и роли с минимальными правами.
\end{itemize}

В будущем планируется реализация:
\begin{itemize}
  \item ограничений по IP;
  \item централизованного управления правами доступа.
\end{itemize}

\customsubsection{Тестирование устойчивости}\label{subsec:testing}
В рамках разработки были проведены следующие виды тестирования:
\begin{itemize}
  \item перезапуск отдельных контейнеров и всей системы в целом;
  \item отключение ключевых компонентов (TextProcessor, Classificator) с последующим восстановлением;
  \item стресс-тестирование системы с симуляцией массового поступления новостей.
\end{itemize}

Результаты подтвердили способность к восстановлению без потерь.
Также было выявлено узкое место при определении тональности новостей.
Из-за того, что обработка происходит поочередно, без применения батчевой обработки, при больших нагрузках (более 50 сообщений в секунды) новости обрабатываюстя медленнее, чем поступают в очередь на предобработку.
Однако стоит отметить, что потери данных не происходит, и после снижения нагрузки состояние системы приходит в нормальное состояние.
