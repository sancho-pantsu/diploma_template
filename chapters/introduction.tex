\newpage
\begin{center}
  \textbf{\large РЕФЕРАТ}
\end{center}

\textbf{Объём работы:}

Выпускная квалификационная работа изложена на {TODO} страницах, содержит {TODO} рисунков, {TODO} таблицы, {TODO} источников, {TODO} приложения.

\textbf{Ключевые слова:}
{TODO}
фильтрационный слой, мониторинг новостей, дедупликация, ключевые слова, анализ тональности, отказоустойчивость, Docker, ElasticSearch, RabbitMQ, MongoDB репликация, мониторинг компонентов, логирование.

\textbf{Текст реферата:}
Выпускная квалификационная работа посвящена разработке и внедрению системы мониторинга новостных ресурсов
с акцентом на реализацию фильтрационного слоя и обеспечению отказоустойчивости компонентов системы.
В работе разработан фильтрационный слой, включающий функции дедупликации новостных сообщений, фильтрации по
ключевым словам и анализа тональности публикаций.
Рассмотрены и реализованы механизмы развертывания компонентов системы в контейнерной среде Docker, с
использованием MongoDB в режиме репликации, RabbitMQ и ElasticSearch.
Предложены методы обеспечения отказоустойчивости и мониторинга компонентов, включая автоматическое
восстановление сервисов и систему логирования событий.
Результаты работы демонстрируют повышение надёжности и отказоустойчивости системы мониторинга новостей,
а также улучшение качества фильтрации данных.
Разработанная система может быть использована для мониторинга репутации в СМИ и социальных сетях,
а также для информационных служб компаний и государственных структур.


\onehalfspacing
\setcounter{page}{2}

\newpage
\renewcommand{\contentsname}{\centerline{\large СОДЕРЖАНИЕ}}
\tableofcontents

\newpage
\begin{center}
  \textbf{\large ВВЕДЕНИЕ}
\end{center}
\addcontentsline{toc}{chapter}{ВВЕДЕНИЕ}


\textbf{Актуальность}

В современном мире, характеризующемся высоким темпом появления и распространения информации, особенно актуальными
становятся задачи мониторинга и анализа новостных ресурсов. Большие объёмы данных, поступающих из различных источников
(RSS-ленты, социальные сети, мессенджеры), требуют использования автоматизированных систем обработки и фильтрации
контента. Одной из ключевых проблем является не только своевременное получение данных, но и их качественная фильтрация
по критериям релевантности, достоверности и значимости для конечного пользователя.

Повышенные требования к надёжности и стабильности таких систем обусловливают необходимость использования
отказоустойчивых решений и развертывания компонентов в распределённой среде. Для этого применяются микросервисные
архитектуры с обеспечением мониторинга и высокой доступности всех сервисов.

\newpage

\textbf{Цель выпускной квалификационной работы} -- разработка фильтрационного слоя системы мониторинга новостных
ресурсов и внедрение микросервисной архитектуры с обеспечением отказоустойчивости её компонентов.

\textbf{Задачи выпускной квалификационной работы:}
\begin{enumerate}
\item Реализация фильтрационного слоя с поддержкой функций:
    \begin{itemize}
    \item дедупликации новостных сообщений;
    \item фильтрации по ключевым словам;
    \item определения тональности публикаций;
    \item определения потенциальной вирусности публикации;
    \end{itemize}
\item Развертывание компонентов системы на базе контейнерной платформы Docker.
\item Обеспечение отказоустойчивости системы на уровне микросервисов и инфраструктуры.
\item Настройка систем логирования.
\item Проведение тестирования отказоустойчивости системы.
\end{enumerate}
