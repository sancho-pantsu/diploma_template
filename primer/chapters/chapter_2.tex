\newpage

\customsection{Постановка задачи}

\customsubsection{Общая характеристика решаемой задачи}
Современные информационные системы ежедневно генерируют огромные объёмы новостных данных.
Одной из ключевых задач является не только оперативный сбор такой информации, но и её эффективная фильтрация с целью предоставления конечным пользователям только актуального, достоверного и релевантного контента.

Решение данной задачи требует реализации высокоэффективных алгоритмов фильтрации и анализа тональности сообщений.
Дополнительно возникает необходимость в обеспечении отказоустойчивости всей системы для её непрерывного функционирования в условиях возможных сбоев отдельных компонентов.

\customsubsection{Цель работы}
Целью настоящей работы является разработка фильтрационного слоя системы мониторинга новостных ресурсов, а также внедрение архитектурных решений, обеспечивающих отказоустойчивую работу компонентов системы.

\customsubsection{Задачи исследования}
В рамках работы решаются следующие задачи:
\begin{enumerate}
    \item Разработать фильтрационный слой, обеспечивающий:
    \begin{itemize}
        \item удаление дублирующихся сообщений (дедупликация);
        \item фильтрацию контента по ключевым словам;
        \item определение тональности публикаций (положительная, отрицательная, нейтральная);
        \item определение потенциальной вирусности публикации.
    \end{itemize}
    \item Развернуть инфраструктуру системы мониторинга новостей, включающую:
    \begin{itemize}
        \item MongoDB в режиме репликации для повышения доступности данных;
        \item RabbitMQ в кластерной конфигурации для обеспечения надёжной передачи сообщений между сервисами;
        \item ElasticSearch для обеспечения поиска и аналитики новостных данных;
        \item Сервисов парсинга, обработки данных, API и Telegram-бот.
    \end{itemize}
    \item Организовать развёртывание сервисов в контейнерной среде Docker для упрощения масштабирования и управления жизненным циклом компонентов.
    \item Реализовать механизмы мониторинга и логирования компонентов системы с целью обеспечения их отказоустойчивости и быстрого обнаружения сбоев.
    \item Провести тестирование системы на предмет надёжности и стабильности функционирования при различных нагрузках.
\end{enumerate}

\customsubsection{Ожидаемые результаты}
Ожидается, что результатом работы станет развернутая система мониторинга новостных ресурсов с реализованным фильтрационным слоем и архитектурой, обеспечивающей отказоустойчивость её основных компонентов.

Система должна демонстрировать высокую устойчивость к сбоям, обеспечивать фильтрацию данных в реальном времени и предоставлять пользователям только актуальную и релевантную информацию.
