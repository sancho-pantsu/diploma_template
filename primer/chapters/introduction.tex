\newpage
{\centering
\normal
\textbf{РЕФЕРАТ}\par
}

Шишкин А.\ Е.\ РАЗРАБОТКА И ВНЕДРЕНИЕ СИСТЕМЫ МОНИТОРИНГА НОВОСТНЫХ РЕСУРСОВ: РЕАЛИЗАЦИЯ ФИЛЬТРАЦИОННОГО СЛОЯ, РАЗВЕРТЫВАНИЕ МИКРОСЕРВИСНОЙ АРХИТЕКТУРЫ И ОБЕСПЕЧЕНИЕ ОТКАЗОУСТОЙЧИВОСТИ КОМПОНЕНТОВ, выпускная квалификационная работа.

Объём работы: 36 страниц, 2 иллюстраций, 14 источников.

Ключевые слова: фильтрационный слой, фильтрация новостей, мониторинг новостей, дедупликация, ключевые слова, анализ тональности, отказоустойчивость, Docker, ElasticSearch, RabbitMQ, MongoDB, репликация, мониторинг компонентов, логирование.

Цель работы: разработка фильтрационного слоя системы мониторинга новостных ресурсов и внедрение микросервисной архитектуры с обеспечением отказоустойчивости её компонентов.

Методы работы: В работе применяются методы построения микросервисных систем, организации отказоустойчивых инфраструктурных решений, алгоритмы обработки текстовых данных и анализа тональности, методы дедупликации и выявления аномалий.
Используются технологии Docker, MongoDB, RabbitMQ, ElasticSearch и языки программирования Python и C\#.

Результат работы: система демонстрирует повышение надёжности и отказоустойчивости мониторинга новостей, а также улучшение качества фильтрации данных.
Разработанная система может быть использована для мониторинга репутации в СМИ и социальных сетях.

Новизна работы: В условиях стремительного роста объёмов информационного потока в СМИ и социальных сетях особенно актуальной становится задача оперативной фильтрации и анализа новостей.
Разработанная система предлагает комплексное решение, объединяющее интеллектуальный фильтрационный слой с микросервисной архитектурой.

\setcounter{page}{2}

\newpage
\renewcommand{\contentsname}{\centerline{\normal СОДЕРЖАНИЕ}}
\begin{spacing}{1.5}
\tableofcontents
\end{spacing}

\newpage
{\centering
\normal
\textbf{ОБОЗНАЧЕНИЯ И СОКРАЩЕНИЯ}\par
}
\addcontentsline{toc}{chapter}{ОБОЗНАЧЕНИЯ И СОКРАЩЕНИЯ}

В настоящей работе применяются следующие обозначения и сокращения:\\
\textbf{APM (Application Performance Monitoring)} — система мониторинга производительности приложений.\\
\textbf{BangerProbability} — вероятность широкой огласки (вирусности) новостного сообщения, определяемая с помощью модели машинного обучения.\\
\textbf{Batched inference} — обработка группы сообщений за один вызов модели.\\
\textbf{CI/CD} — непрерывная интеграция и доставка, автоматизирующая процесс сборки и деплоя.\\
\textbf{Classificator} — микросервис, оценивающий вирусный потенциал новостей (BangerProbability).\\
\textbf{Docker Swarm} — встроенный оркестратор контейнеров Docker, применяемый для автоматизированного масштабирования и управления микросервисами.\\
\textbf{Docker} — платформа контейнеризации, используемая для изоляции и развёртывания компонентов системы.\\
\textbf{ElasticSearch} — полнотекстовый поисковый движок, применяемый для индексирования новостей и логов.\\
\textbf{FastAPI} — веб-фреймворк для создания REST API на Python, применяемый в сервисах TextProcessor и Classificator.\\
\textbf{Filebeat} — агент для сбора логов из контейнеров и их отправки в ElasticSearch.\\
\textbf{Filtrator} — основной фильтрационный микросервис, реализующий все этапы обработки новостей.\\
\textbf{Healthcheck} — механизм автоматической проверки «здоровья» контейнеров.\\
\textbf{Kibana} — инструмент визуализации данных из ElasticSearch (в частности — логов и метрик).\\
\textbf{MongoDB} — документо-ориентированная база данных, используемая для хранения новостей, метаинформации и пользовательских настроек.\\
\textbf{Named Entity Recognition (NER)} — извлечение именованных сущностей из текста (планируется внедрение).\\
\textbf{NewsFilterService} — компонент для семантического сравнения новостей при дедупликации.\\
\textbf{NewsToSend} — коллекция в MongoDB, содержащая новости, готовые к отправке пользователям.\\
\textbf{REST API} — архитектурный стиль взаимодействия между компонентами системы по протоколу HTTP.\\
\textbf{RabbitMQ} — брокер сообщений, обеспечивающий асинхронное взаимодействие между микросервисами системы.\\
\textbf{Retry policy (Политика повторных попыток)} — механизм обработки временных сбоев во взаимодействии сервисов.\\
\textbf{Telegram-бот (TG-бот)} — компонент, осуществляющий доставку отобранных новостей пользователям через мессенджер Telegram.\\
\textbf{TextProcessor} — микросервис, отвечающий за векторизацию текста и определение его эмоциональной окраски.\\
\textbf{Vectorization} — процесс преобразования текста в вектор признаков с помощью предобученных моделей.\\
\textbf{WireGuard} — VPN-протокол, обеспечивающий безопасное соединение между сервером и внешними узлами.\\
\textbf{Анализ тональности (Sentiment Analysis)} — метод обработки текстов, определяющий эмоциональную окраску сообщения (положительную, отрицательную или нейтральную).\\
\textbf{Дедупликация} — процесс выявления и удаления повторяющихся или семантически схожих новостных сообщений.\\
\textbf{Ключевые слова} — заранее определённые слова или фразы, служащие фильтром релевантного новостного контента.\\
\textbf{Репликация MongoDB} — механизм обеспечения отказоустойчивости и доступности данных путём создания копий базы данных.\\
\textbf{Фильтрационный слой} — компонент системы, выполняющий интеллектуальный отбор, анализ и обработку новостных данных по заданным критериям (дедупликация, анализ тональности, фильтрация по ключевым словам).

\newpage
{\centering
\normal
\textbf{ВВЕДЕНИЕ}\par
}
\addcontentsline{toc}{chapter}{ВВЕДЕНИЕ}

\textbf{Актуальность}

В современном мире, характеризующемся высоким темпом появления и распространения информации, актуальными становятся задачи мониторинга и анализа новостных ресурсов.
Большие объёмы данных, поступающих из различных источников (RSS-ленты, социальные сети, мессенджеры), требуют использования автоматизированных систем обработки и фильтрации контента.
Одной из ключевых проблем является не только своевременное получение данных, но и их качественная фильтрация по критериям релевантности, достоверности и значимости для пользователя.

Повышенные требования к надёжности и стабильности таких систем обусловливают необходимость использования отказоустойчивых решений и развертывания компонентов в распределённой среде.
Для этого применяются микросервисные архитектуры с обеспечением мониторинга и высокой доступности всех сервисов.

\textbf{Цель выпускной квалификационной работы} -- разработка фильтрационного слоя системы мониторинга новостных
ресурсов и внедрение микросервисной архитектуры с обеспечением отказоустойчивости её компонентов.

\textbf{Задачи выпускной квалификационной работы:}
\begin{enumerate}
\item Реализация фильтрационного слоя с поддержкой функций:
    \begin{itemize}
    \item дедупликации новостных сообщений;
    \item фильтрации по ключевым словам;
    \item определения тональности публикаций;
    \item определения потенциальной вирусности публикации;
    \end{itemize}
\item Развертывание компонентов системы на базе контейнерной платформы Docker.
\item Обеспечение отказоустойчивости системы на уровне микросервисов и инфраструктуры.
\item Настройка систем логирования.
\item Проведение тестирования отказоустойчивости системы.
\end{enumerate}

\textbf{Объект исследования} — система мониторинга новостных ресурсов и потоков данных из различных источников.\\
\textbf{Предмет исследования} — архитектура отказоустойчивой системы обработки данных с реализацией фильтрационного слоя.

\textbf{Методы исследования}\\
В работе применяются методы построения микросервисных систем, организации отказоустойчивых инфраструктурных решений, алгоритмы обработки текстовых данных и анализа тональности, методы дедупликации и выявления аномалий.
Используются технологии Docker, MongoDB, RabbitMQ, ElasticSearch и языки программирования Python и C\#.

\textbf{Практическая значимость}\\
Результатом исследования является внедрённая система мониторинга новостей, обеспечивающая высокую отказоустойчивость и достоверность фильтрации информации.
Решение может быть применено в системах корпоративной аналитики, службах мониторинга репутации и СМИ, а также в государственных информационных системах.

\textbf{Структура работы}\\
Работа состоит из введения, трёх глав, заключения, списка использованных источников и приложений.
\begin{enumerate}
    \item В первой главе рассматриваются существующие подходы к мониторингу новостей и фильтрации данных, а также архитектурные решения для построения отказоустойчивых систем.
    \item Во второй главе выполняется постановка задачи
    \item Во третьей главе описывается реализация фильтрационного слоя новостей.
    \item В четвертой главе приведены решения по развертыванию компонентов системы и обеспечению их отказоустойчивости.
\end{enumerate}
